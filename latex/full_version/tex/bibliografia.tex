\begin{thebibliography}{9}
\bibitem{robocup}
	Oficjalna strona Ligi \textit{RoboCup} dostępna pod adresem:\\
	\url{www.robocup.org}
	
\bibitem{RRT}
J. Bruce, M. Veloso:
\emph{Real-Time Randomized Path Planning for Robot Navigation}. Carnegie Mellon University, 2002.

\bibitem{RRT2}
J.Kim J.M. Esposito, V. Kumar:
\emph{An RRT-Based algorithm for testing and validating multi-robot controllers}. University of Pennsylvania, US Naval Academy, 2005.

\bibitem{duleba}
	I. Dulęba:
	\emph{Metody i algorytmy planowania ruchu robotów mobilnych i manipulacyjnych}.
	Warszawa, Akademicka Oficyna Wydawnicza EXIT, 2001.

\bibitem{CVM_1}
	T. Quasn, L.Pyeatt, J.Moore:
	\emph{Curvature-Velocity Method for Differentially Steered Robots}.
	AI Robotics Lab
	Computer Science Department,
	Texas Tech University, 2003.

\bibitem{CVM_2}
	R. Simmons:
	\emph{The Curvature-Velocity Method for Local Obstacle Avoidance}.
	School of Computer Science,
	Carnegie Mellon University, 1996.
\bibitem{majchrowski}
	M. Majchrowski:
	\emph{Algorytm unikania kolizji przez robota mobilnego bazujący na przeszukiwaniu
	przestrzeni prędkości}.
	Praca Magisterska, Politechnika Warszawska, 2006.

\bibitem{VFH_1}
	J. Borenstein, Y. Koren:
	\emph{ The vector field histogram -- fast obstacle avoidance for mobile robots}.
	IEEE Transaction on Robotics and Automation, 1991.

\bibitem{VFH_2}
	J. Borenstein,Y. Koren:
	\emph{Histogramic in-motion mapping
	for mobile robot obstacle avoidance}.
	Department of Mechanical Engineering and Applied Mechanics,
	The University of Michigan, 1991.

\bibitem{dynamic_window}
	D. Fox, W. Burgard, S. Thrun:
	\emph{The Dynamic Window Approach to Collision Avoidance}.
	Department of Computer Science, University of Bonn;
	Department of Computer Science, Carnegie Mellon University, Pittsburgh, 1991.
		
 \bibitem{stp}
	B.Browning, J.Bruce, M.Bowling, M.Veloso:
	\emph{STP: Skills, tactics and plays for multi-robot control
             in adversarial environments}.
	  Carnegie Mellon University, Pittsburgh. 2004.

 \bibitem{omni_base_1}
	R.Rojas, A.Gloye F\"{o}orster:
	\emph{ Holonomic Control of a robot with an omnidirectional
	      drive}.
	  Freie Universität Berlin, 2006.

\bibitem{omni_base_2}
	Marcelo H. Ang Jr.,Ir. M. Steinbuch:
	\emph{Design of an omnidirectional universal mobile platform}.
	  Eindhoven University of Technology, 2005.	  
\bibitem{dribbling}
	X. Li, M.Wang, A.Zell:
	\emph{ Dribbling control of omnidirectional soccer robots}.
	In Proceedings of 2007 IEEE International Conference on Robotics and Automation (ICRA'07).

\bibitem{inzynierka}
	M.Gąbka, K.Muszyński
	Praca dyplomowa inzynierska
	\emph{Środowisko symulacyjne i algorytm unikania kolizji robota mobilnego
	grającego w piłkę nożną}.
	Politechnika Warszawska, 2008.

\bibitem{trapezy1}
	J.Bruce, M.Bowling, B.Browning, M.Veloso:
	\emph{Multi-Robot Team Response to a Multi-Robot Opponent Team}.
	Carnegie Mellon University, Pittsburgh, 2002.

\bibitem{trapezy2}
	J.Bruce, M.Veloso:
	\emph{Real-time multi-robot motion planning with safe dynamics}.
	Carnegie Mellon University, Pittsburgh, 2006.

\bibitem{navi_func1}
	D. E. Koditschek, and E.Rimon:
	\emph{Robot navigation functions on manifolds with boundary}.
	Advances in Applied Mathematics
	Volume 11 Issue 4, Dec. 1990.

\bibitem{navi_func2}
	D. E. Koditschek, and E.Rimon:
	\emph{Exact robot navigation using artificial potential functions}.
	IEEE Transactions on Robotics and Automation.
	Vol. 8, no. 5, pp. 501-518. Oct. 1992
\bibitem{ISR}
	C. Zieliński, W. Szynkiewicz:
	\emph{Konspekt do wykładu: Inteligentne Systemy robotyczne}.
	Politechnika Warszawska, 2008.
\bibitem{gazebo_experts}
	N. Koenig:
	\emph{Gazebo. The Instant Expert's Guide}.
	Player Summer School on Cognitive Robotics, Monachium 2007.
\bibitem{hamada_mgr}
 M.Hamada
 \emph{Układ sterowania autonomicznym robotem mobilnym}. Praca magisterska, Politechnika Warszawska, 2007.
\end{thebibliography}
