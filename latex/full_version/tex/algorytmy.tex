\chapter{Algorytmy unikania kolizji \label{chap:algorytmy}}
 W rozdziale zostanie szerzej zaprezentowany jeden z algorytmów unikania kolizji jakim jest RRT(\texttt{Rapidly-Exploring Random Tree}).
Poruszona zostanie także kwestia powodów, dla których wybrano tę, a nie inną metodę. Uzasadnione zostanie odejście od algorytmu opracowanego w ramach pracy 
inżynierskiej (CVM \textit{Curvature Velocity Method}). Ponadto opisane zostaną inne metody planowania ścieżki oraz omówione zostaną ich właściwości. Na tej podstawie zostanie 
uzasadniony wybór algorytmu RRT.

\section{Krótki przegląd algorytmów unikania kolizji}
Jednym z podstawowych problemów podczas poruszania się każdej jednostki mobilnej jest wyznaczenie bezkolizyjnej ścieżki prowadzącej do celu.
Współczesna robotyka zna wiele algorytmów rozwiązujących z mniejszym bądź większym sukcesem to zadanie. Użycie wielu z nich jest jednak w pełni uzasadnione tylko w
szczególnych okolicznościach. Poniżej zostaną zaprezentowane najbardziej znane algorytmy unikania kolizji (omówione także w pracy inżynierskiej \cite{inzynierka}).
\subsection{Algorytm Bug}
	Najprostszym algorytmem, służącym do wyznaczania bezkolizyjnej ścieżki jest algorytm \emph{Bug}. Jak sama nazwa wskazuje zasada jego działania wzorowana
	jest na zachowaniu pluskwy. W momencie, gdy podążający do celu robot napotka przeszkodę, powinien okrążyć ją całkowicie i zapamiętać pozycję, w której znajdował
	się najbliżej celu. W kolejnym kroku, robot powinien ponownie okrążyć przeszkodę, z tym wyjątkiem, że jeśli osiągnie uprzednio zapamiętaną pozycję, powinien oderwać się od przeszkody
	i rozpocząć poruszanie w kierunku właściwego celu. Przykładowe rozwiązanie otrzymane za pomocą algorytmu \emph{Bug} zaprezentowano na rysunku \ref{fig:Bug1}.
	Stosowanie tej metody wymaga wyposażenie robota w następujące czujniki:
	\begin{itemize}
	\item czujnik celu -- wyznaczający kierunek w stronę celu oraz umożliwiający pomiar odległości do celu,
	\item czujnik lokalnej widoczności -- umożliwiający  podążanie wzdłuż konturu przeszkody. 
	\end{itemize}
	\begin{figure}[h]
	\centering
	\includegraphics[scale=0.7]{./algorytmy/Bug1}
	\caption{ Bezkolizyjna ścieżka wyznaczona przez algorytm Bug }\label{fig:Bug1}
	\end{figure}
	Dużą wadą algorytmu jest fakt, że otrzymana ścieżka jest daleka od optymalnego rozwiązania. Sterowany robot zmuszony jest do całkowitego okrążenia przeszkody, niezależnie 
	od obranego wstępnie kierunku ruchu. Co więcej, w niektórych sytuacjach robot nie jest w stanie okrążyć przeszkody (przykładem może być ściana korytarza). 
	W takich sytuacjach algorytm nie może zostać wykorzystany.
	Podstawową wersję algorytmu można zmodyfikować, tak aby poprawić efektywność. Podczas okrążania przeszkody robot może sprawdzać czy punkt w którym aktualnie się znajduje jest poszukiwanym punktem położonym najbliżej celu. 
	Przykładowa ścieżka wyznaczona za pomocą rozszerzonej wersji algorytmu zamieszczona została na rysunku \ref{fig:Bug2}.
	Robot otacza przeszkodę w wybranym wcześniej kierunku i odłącza się od niej w momencie przecięcia
 	prostej łączącej punkt startowy z punktem docelowym. Uzyskana w ten sposób ścieżka nadal zależy od wybranego a~priori kierunku jazdy.	
	\begin{figure}
	\centering
	\includegraphics[scale=0.7]{./algorytmy/Bug2}
	\caption{Zasada działania algorytmu Bug w wersji rozszerzonej}\label{fig:Bug2}
	\end{figure}
	Algorytm Bug w żadnej ze swoich wersji nie uwzględnia ograniczeń wynikających z budowy robota. Jest to jeden z najprostszych algorytmów służących do unikania
	kolizji, jednak jego efektywność jest bardzo mała. Wyznaczona ścieżka jest daleka od optymalnej, a podczas okrążania przeszkody robot nie może poruszać się z duża prędkością,
	co wydłuża czas dojazdu do celu.
\subsection{Algorytm VFH}
	Pełna  anglojęzyczna nazwa metody brzmi \mbox{ \emph{Vector Field Histogram}}. Na język polski można ją
	przetłumaczyć jako algorytm histogramu pola wektorowego. Należy ona do grupy metod, które w czasie
	rzeczywistym pozwalają na jednoczesne wykrywanie i omijanie przeszkód oraz kierowanie robota na cel.
	Algorytm ten został szczegółowo opisany w pracy inżynierskiej \cite{inzynierka}, więcej informacji można też znaleźć w publikacjach autorów (\cite{VFH_1} oraz \cite{VFH_2}).
	W skrócie, jego zasada działania opiera się na konstrukcji dwuwymiarowego opisu świata w postaci siatki. Każdemu elementowi siatki przypisany jest poziom ufności oddający prawdopodobieństwo
	z jakim w danym położeniu może pojawić się przeszkoda (rysunek \ref{fig:VFH_siatka}).
	\begin{figure}[H]
	\centering
	\includegraphics[scale=0.44]{./algorytmy/VFH_siatka}
	\caption{ Zasada tworzenia dwuwymiarowego histogramu \newline(na podstawie \cite{VFH_2})}\label{fig:VFH_siatka}
	\end{figure}
	Tak skonstruowana mapa redukowana jest do jednowymiarowego histogramu biegunowego (rysunek \ref{fig:VFH_hist_biegunowy}), który dodatkowo wygładzany jest przez
	filtr dolnoprzepustowy. W ten sposób otrzymuje się informację o poziomie ufności wystąpienia przeszkody poruszając się w danym kierunku (świadomie rezygnuje się z informacji o odległości od przeszkody).
	Na podstawie tego histogramu wyznaczane są sterowania dla robota. Histogram jest analizowany w poszukiwaniu minimów lokalnych, wszystkie doliny, których poziom ufności
	znajduje się poniżej ustalonego umownie progu są rozważane przy wyznaczaniu kierunku jazdy robota. Jeśli minimów jest kilka wybierane jest to, którego kierunek prowadzi najbliżej
	do celu. 
	Algorytm do poprawnego działania  potrzebuje informacji o rozłożeniu przeszkód w otoczeniu robota. W oryginalnej implementacji była ona pozyskiwana z sensorów ultradźwiękowych.
	Można je również zastąpić z powodzeniem czujnikami laserowymi. Obraz video z kamery umieszczonej nad eksplorowanym światem także dostarcza tej informacji (jak w rozgrywkach \emph{Small-size League}).
	\begin{figure}[H]
	\centering
	\includegraphics[scale=0.5]{./algorytmy/VFH_hist_biegunowy}
	\caption{ Histogram biegunowy dla przykładowego rozmieszczenia przeszkód \newline(na podstawie \cite{ISR})}\label{fig:VFH_hist_biegunowy}
	\end{figure}

	\subsection{Technika dynamicznego okna} \label{subsec:dynamic_window}
	Popularną, stosowaną w robotyce metodą unikania kolizji jest także technika dynamicznego okna. Algorytm ten analizuje jedynie możliwe do osiągnięcia w danej sytuacji prędkości.
	Więcej informacji na temat algorytmu można znaleźć w publikacji \cite{dynamic_window}.
	Ujmując w jednym zdaniu algorytm polega na przeszukiwaniu zbioru dopuszczalnych prędkości liniowych i kątowych, tak aby maksymalizować funkcję celu. Sposób konstrukcji funkcji celu
	gwarantuje, że wybrane prędkości będą prowadzić robota w zamierzonym kierunku oraz, że nie natrafi na przeszkodę.
\subsection{Algorytmy pól potencjałowych}
\subsubsection{Algorytm w wersji podstawowej}
	Kolejne podejście do problemu nawigacji robota mobilnego zaczerpnięto z fizyki. Problem znalezienia bezkolizyjnej ścieżki został sprowadzony do problemu konstrukcji
	funkcji opisującej rozkład energii w danym środowisku. W takim układzie dotarcie do celu jest równoważne ze znalezieniem minimum funkcji opisującej rozkład sztucznego pola
	potencjałowego. Robot podąża w kierunku ujemnego gradientu tej funkcji, mamy zatem $c(t)=- \nabla U(c(t))$. W klasycznym podejściu sztuczne pole potencjałowe tworzy
	się w ten sposób, że przeszkody są źródłem ujemnego pola(odpychającego), natomiast cel emituje pole dodatnie. Ujęte jest to w następujących wzorach:
	\begin{equation}
	U(q) = U_{att}(q) + U_{rep}(q)
	\end{equation}
	\begin{equation}\label{eq:U_att}
		U_{att}(q) = \left\{ 
		\begin{array}{l l}
		\frac{1}{2}\xi d^2 & \quad \mbox{dla $d\leqq d^{*}_{goal}$ }\\
		\xi d^{*}_{goal}d - \frac{1}{2}(d^{*}_{goal})^{2} & \quad \mbox{dla $d\geqq d^{*}_{goal}$ }\\
		\end{array} \right. 
	\end{equation}
	gdzie $d=d(q,q_{goal})$ jest odległością danego położenia od celu, $\varepsilon$ jest stałą określającą poziom przyciągania do celu, natomiast $d^{*}_{goal}$ określa próg,
	po którym funkcja z kwadratowej przechodzi w trójkątną.
	Oddziaływanie przeszkód opisane jest następującym wzorem:
	\begin{equation}
		U_{rep}(q) = \left\{ 
		\begin{array}{l l}
		\frac{1}{2} \eta{ ( \frac{1}{D(q)} - \frac{1}{ Q^{*} } )}^2  & \quad \mbox{dla $D(q)\leq Q^{*} $ }\\
		0 & \quad \mbox{dla $D(q)> Q^{*}$ }\\
		\end{array} \right.
	\end{equation}
	gdzie $ Q^{*} $ jest zasięgiem pola odpychającego przeszkody, natomiast $\eta$ odpowiada za siłę tego pola. 
	Kierunek, w którym podążać ma robot wyznaczany jest ze wzoru:
	\begin{equation}
	- \nabla U(q) = - \nabla U_{att}(q) - \nabla U_{rep}(q)
	\end{equation}
	Powyższe podejście w stosunkowo prosty sposób umożliwia wyznaczenie kierunku bezkolizyjnej ścieżki do celu. Posiada ono jednak pewną dość istotną wadę. Mianowicie w pewnych, 
	szczególnych sytuacjach robot może utknąć w minimum lokalnym. Sposób w jaki konstruowana jest funkcja celu w żaden sposób nie wyklucza występowania minimów lokalnych.
	W przypadku, gdy sterowany robot utknie w takim minimum, konieczna jest reakcja ze strony wyższej warstwy nawigującej maszyną. Przykładowo można wykorzystać algorytm 
	błądzenia losowego.
\subsubsection{Funkcja nawigacji \label{subsubsec:navigation_func}}
	Istnieje jednak pewna szczególna funkcja, która posiada tylko globalne minimum, została ona zdefiniowana w publikacjach \cite{navi_func1} oraz \cite{navi_func2}.
	Przykładowy wykres funkcji nawigacji zamieszczony jest na rysunku ~\ref{fig:func_navi}. Zasięg oddziaływania minimum globalnego zależy od kilku istotnych parametrów, które zawarte są 
	we wzorze opisującym rozkład sztucznego potencjału.
	\begin{figure}[!h]
	\centering
	\includegraphics[scale=0.3]{./algorytmy/navigation_func_k_10}
	\caption{Funkcja nawigacji dla $\kappa=10$}
	\label{fig:func_navi}
	\end{figure}
	Dla bieżącego położenia robota $q$ funkcja nawigacji zdefiniowana jest następująco:
	\begin{equation}
	\label{eq:navi_func}
	\phi = \frac{(d(q,q_{goal}))^2}{ (\lambda \beta(q) + d(q,q_{goal})^{2\kappa} )^{\frac{1}{\kappa}}}
	\end{equation}
	gdzie $\beta(q)$ jest iloczynem funkcji odpychających zdefiniowanych dla wszystkich przeszkód:
	\begin{equation}
	\beta  \stackrel{ \vartriangle }{ = } \prod_{i=0}^{N}{ \beta_i(q)}    
	\end{equation}
	Natomiast dla każdej przeszkody($i>0$) funkcja określona jest następująco:
	\begin{equation}
	\beta_{i}(q)=(d(q,q_i))^2 - r_i^{2} \text{ dla i = 1...N, gdzie N jest liczbą przeszkód}
	\end{equation}
	Powyższą funkcję definiuje się dla każdej z przeszkód o promieniu $r_i$, której środek znajduje się w punkcie $q_i$.
	Jak łatwo zauważyć, funkcja ta przyjmuje ujemne wartości wewnątrz okręgu opisującego przeszkodę, natomiast dodatnie na zewnątrz okręgu.
	Dodatkowo definiuje się funkcję $\beta_{0}(q)= -(d(q,q_0))^{2} + r_0^{2}$, w której parametry $r_0$ oraz $q_0$ oznaczają odpowiednio promień świata w którym porusza się robot
	oraz środek tego świata.
	Wzór \ref{eq:navi_func} posiada dwa istotne parametry. Pierwszy z nich, $\lambda$ ogranicza przeciwdziedzinę do przedziału $[0,1]$, gdzie wartość $0$ jest tożsama z osiągnięciem celu, natomiast
	$1$ jest osiągana na brzegu każdej z przeszkód. Drugi parametr $\kappa$ odpowiednio dobrany powoduje, że blisko celu wykres funkcji $\phi$ przybiera kształt misy.
	Zwiększanie parametru $\kappa$ powoduje przesuwanie globalnego minimum w kierunku położenia punktu docelowego. Zwiększone jest zatem oddziaływanie przyciągającego pola emitowanego przez 
	punkt docelowy.

	Wykres funkcji nawigacji przedstawiony na rysunku \ref{fig:func_navi} został sporządzony dla następujących parametrów (wszystkie jednostki wyrażone są w metrach):
	\begin{enumerate}
	\item eksplorowany świat ograniczony jest okręgiem o środku w punkcie $q_0(2.7;3.7)$ i promieniu $r_0 = 7.4$,
	\item $\lambda = 0.2$,
	\item $\kappa = 10$,
	\item przeszkody umiejscowione są w punktach $(2;2),(3;3)(4;4),(3.5;3.5)$ i mają stały promień, odpowiadający modelowi robota
	zastosowanemu podczas doświadczeń $r_i = 0.14 $,
	\item punkt docelowy ma współrzędne $q_g(2.7;0.675)$,
	\item funkcja nawigacji została wykreślona z krokiem $0.1$.
	\end{enumerate}

\subsection{Algorytm CVM (\textit{Curvature Velocity Method})}
	W pracy inżynierskiej \cite{inzynierka} jako docelowy algorytm unikania kolizji wybrany został właśnie \texttt{CVM}.
	Główną motywacją wykorzystania tej metody była wykorzystywana wtedy różnicowa baza jezdna. Duża zaletą algorytmu są także niskie nakłady obliczeniowe oraz prosty sposób
	sterowania zasięgiem algorytmu. Metodę można także łatwo zmodyfikować tak, aby uwzględniała dynamikę przeszkód znajdujących się w danym środowisku.

	Metoda krzywizn i prędkości (ang. \textit{Curvature Velocity Method}) została zaproponowana przez R. Simmonsa w
	\cite{CVM_2}, należy ona do grupy metod bazujących na przeszukiwaniu przestrzeni prędkości, a jej działanie
	jest zbliżone do techniki dynamicznego okna zaprezentowanej w \ref{subsec:dynamic_window}.

	Spośród wszystkich par $(v,\omega)$ dozwolonych prędkości liniowych i kątowych, wybierana jest taka, która osiąga największą wartość funkcji celu.
	Podejście takie umożliwia uwzględnienie przede wszystkim ograniczeń kinematycznych, ale także i dynamicznych.
	Wyznaczone przez algorytm sterowanie na następny krok definiuje łuk $c=\frac{\omega}{v}$, po którym będzie
	poruszał się robot do chwili wyznaczenia kolejnego sterowania.
	
	Algorytm korzysta z funkcji celu, której konstrukcja ma zapewnić, że wyznaczone prędkości prowadzą robota do celu, ale jednocześnie zapewniają szybki dojazd do zadanego punktu
	oraz unikanie kolizji.
	Takie zachowanie osiągnięto definiując funkcję celu jako sumę ważoną następujących trzech składowych:
	\begin{equation} \label{eq:fcelu_cvm}
	F(v,\omega)=\alpha_1 \mathcal{V}(v,\omega)+\alpha_2 \mathcal{D}(v,\omega) + \alpha_3 \mathcal{G} (v,\omega))
	\end{equation}
	gdzie:
	\begin{itemize}
	 \item $\alpha_1, \alpha_2, \alpha_3$ współczynniki wagowe,
	 \item funkcja $\mathcal{V}(v,\omega)$ określa stosunek między ocenianą prędkością liniową a maksymalną,
	 \item funkcja $\mathcal{D}(v,\omega)$ określa odległość od najbliższej przeszkody na którą napotka robot poruszający się po trajektorii wyznaczonej przez $(v,\omega)$,
	 \item funkcja $\mathcal{G}(v,\omega)$ odpowiada za kierowanie się robota na cel.
	\end{itemize}
	Zmiana wartości współczynników wagowych, wpływa bezpośrednio na zmianę zachowania sterowanego robota. Przykładowo, jeśli $\alpha_2$ będzie równe zero, robot nie będzie unikał kolizji.
	Kluczem do poprawnego działania tej metody jest wyznaczenie odpowiednich wartości współczynników.
	Zachowaniem robota można sterować poprzez zmianę wartości współczynników wagowych. W dalszej części rozdziału zamieszczono szczegółowy opis poszczególnych składowych funkcji celu
	oraz zaprezentowano zasadę działania algorytmu.

	\section{Opis działania CVM \label{sec:CVM:action}} 
	Omawianie mechanizmu wyboru najlepszego sterowania z wykorzystaniem algorytmu CVM należy rozpocząć od przypomnienia funkcji celu (\ref{eq:fcelu_cvm})
	\begin{equation*}
	F(v,\omega)=\alpha_1 \mathcal{V}(v,\omega)+\alpha_2 \mathcal{D}(v,\omega) + \alpha_3 \mathcal{G} (v,\omega))
	\end{equation*}
	Składowe odpowiadające za kierowanie robota na cel oraz za maksymalizację prędkości liniowej definiowane są następująco:
	\begin{gather}
	\mathcal{V}(v,\omega)=\frac{v}{V_{max}} \label{eq:CVM_speed}\\
	\mathcal{G}(v,\omega)=1-\frac{|\theta_{cel} -T_{alg}\omega|}{\pi} \label{eq:CVM_head}
	\end{gather}
	gdzie:
	\begin{itemize}
	 \item $V_{max}$ -- jest maksymalną prędkością liniową, którą może osiągnąć robot
	 \item $T_{alg}$ -- określa odstęp pomiędzy kolejnymi uruchomieniami algorytmu
	 \item $\theta_{cel}$ -- jest orientacją do celu w układzie współrzędnych związanym ze sterowanym robotem 
	\end{itemize}
	Zdefiniowanie funkcji odpowiedzialnej za unikanie kolizji jest zadaniem dużo trudniejszym. Na wstępie określone zostaną dodatkowe funkcje wykorzystywane w CVM, służące do 
	przekształcenia opisu przeszkód we współrzędnych kartezjańskich do przestrzeni prędkości liniowej i kątowej robota. Dla danej przeszkody $o$ wyznaczana jest odległość do punktu $p$, w którym wystąpi
	kolizja z przeszkodą, jeśli robot będzie poruszał się po trajektorii $c=\frac{\omega}{v}$. Wyznaczona w ten sposób wartość oznaczana jest symbolem
	$d_c((0,0),p)$, gdzie $p$ jest punktem zderzenia z przeszkodą. Dla danego zestawu prędkości, można zatem zdefiniować funkcję określającą odległość robota od przeszkody:
 	\begin{equation}\label{eq:CVM_dv}
	d_v(v,\omega,p)= \left\{ 
	\begin{array}{l l}
  	d_c((0,0),p_i & \quad \mbox{dla $v\neq0$}\\
  	\infty & \quad \mbox{dla $v=0$ }\\
	\end{array} \right. 
	\end{equation}
	W rzeczywistych sytuacjach najczęściej algorytm ma do czynienia z pewnym zbiorem przeszkód $O$. W takiej sytuacji, zawsze wybierana jest odległość do najbliższej przeszkody leżącej na trajektorii wyznaczonej
	przez zestaw prędkości. Dla danego zbioru przeszkód $O$ funkcję określającą odległość od przeszkody można zatem zapisać następująco:
 	\begin{equation}\label{eq:CVM_Dv}
	D_v(v,\omega,O)= \underset{o \in O}{\operatorname{inf}} d_v(v,\omega,o)
	\end{equation}
	Zazwyczaj jednak informacja o położeniu przeszkód jest ograniczona do pewnego obszaru. Może wynikać on na przykład z zasięgu zastosowanych czujników lub z celowego ograniczenia
	zasięgu działania algorytmu w przypadku, gdy możliwy jest dostęp do globalnej informacji o położeniu przeszkód. Celowe ograniczenie zasięgu algorytmu może przyspieszyć jego działanie.
	W przypadku silnie dynamicznego środowiska, uwzględnianie odległych przeszkód może okazać się zbędne. W dalszej części pracy zasięg algorytmu  będzie oznaczany jako $L$.
	Uwzględniając powyższe rozważania, funkcja  (\ref{eq:CVM_Dv}) jest zatem ograniczona od góry w sposób następujący:
 	\begin{equation}\label{eq:CVM_Dv_ogr}
	D_{ogr}(v,\omega,O)= \min(L, D_v(v,\omega,O))
	\end{equation}
	Po uprzednim określeniu elementarnych funkcji,  można zdefiniować postać funkcyjną składowej funkcji celu odpowiadającej w algorytmie za unikanie kolizji:
	\begin{equation} \label{eq:CVM_dist}
	\mathcal{D}(v,\omega)=\frac{D_{ogr}}{L}
	\end{equation}
 	
	Aby sposób obliczania odległości do przeszkody był możliwie jak najprostszy, każda przeszkoda jest
	reprezentowana w algorytmie jako okrąg o  promieniu \emph{r} (wynikającym z jej rozmiarów) i współrzędnych środka $(x_0,y_0)$. Można zatem dość łatwo wyznaczyć odległość, jaką przebędzie robot poruszający się po łuku wyznaczonym przez parę $(v,\omega)$ do
	momentu przecięcia z okręgiem. Zostało to zobrazowane na rysunku \ref{fig:CVM_dist}.
	\begin{figure}[!h]
	\centering
	%\includegraphics[scale=0.43]{./algorytmy/CVM_dist}
	\includegraphics[scale=0.35]{./algorytmy/CVM_dist_new}
	\caption{ Obliczanie odległości od przeszkody \label{fig:CVM_dist}}
	\end{figure}
	
	Dla robota o początkowej orientacji zgodnej z osią $OY$, poruszającego się po trajektorii o krzywiźnie $c$ przecinającej
	okrąg opisujący przeszkodę w punkcie $P(x,y)$ prawdziwe są następujące wzory:
	\begin{equation}\label{eq:CVM_teta}
	\theta= \left\{ 
	\begin{array}{l l}
	\text{atan2}(y,x-\frac{1}{c} )& \quad \mbox{dla $c<0$}\\
  	\pi-\text{atan2}(y,x-\frac{1}{c}) & \quad \mbox{dla $c>0$}\\
	\end{array} \right. 
	\end{equation}
	
	\begin{equation}
	d_c((0,0),P)= \left\{ 
	\begin{array}{l l}
  	y& \quad \mbox{dla $c=0$}\\
  	|\frac{1}{c}|\theta & \quad \mbox{dla $c\neq0$}\\
	\end{array} \right. 
	\end{equation}
	Zastosowana we wzorze (\ref{eq:CVM_teta}) funkcja \emph{atan2} zwraca wartości z przedziału $[-\pi;\pi]$, zatem uwzględnia w której ćwiartce leży kąt:
	\begin{equation}\label{eq:atan2}
	\text{atan2}(y,x)= \left\{ 
	\begin{array}{l l}
	\arctan{\frac{|y|}{|x|}} \sgn{(y)} & \quad \mbox{dla $x>0$}\\
	\frac{\pi}{2}\sgn{(y)} & \quad \mbox{dla $x=0$}\\
	\pi-\arctan{\frac{|y|}{|x|}} \sgn{(y)} & \quad \mbox{dla $x<0$}\\
	\end{array} \right. 
	\end{equation}
	
	Jednakże obliczanie odległości od przeszkody dla każdego sterowania $(v,\omega)$ w sposób omówiony powyżej
	jest czasochłonne. Stąd konieczne jest kolejne uproszczenie. Łatwo zauważyć, że z punktu $(0,0)$ w którym
	znajduje się robot można poprowadzić dwa łuki styczne do okręgu opisującego przeszkodę $p$. Zatem na zewnątrz
	krzywych stycznych odległość  $d_c((0,0),p)$ jest nieskończona. Natomiast dla krzywizn zawierających się
	pomiędzy łukami stycznymi można dokonać przybliżenia wartością stałą. Omawiana sytuacja została zobrazowana na
	rysunku \ref{fig:CVM_styczne}.
	\begin{figure}[!h]
	\centering
	%\includegraphics[scale=0.45]{./algorytmy/CVM_styczne}
	\includegraphics[scale=0.32]{./algorytmy/CVM_styczne_new}
	\caption{ Zasada wyznaczania przedziału $(c_{min}$,$c_{max})$ \label{fig:CVM_styczne}}
	\end{figure}
	Wyznaczenie okręgów stycznych do okręgu opisującego przeszkodę sprowadza się do znalezienia takiego promienia $r$, dla którego
	poniższy układ równań ma dokładnie jedno rozwiązanie:
	\begin{equation}\label{eq:CVM_styczne}
	 \left\{ 
	\begin{array}{l l}
  	(x-x_0)^2 + (y-y_0)^2={r_0}^2\\
  	 (x-r)^2 + y^2={r}^2\\
	\end{array} \right. 
	\end{equation}
	gdzie $(x_0,y_0)$ jest środkiem okręgu opisującego przeszkodę, natomiast $r_0$ jego promieniem.
	W wyniku takiego postępowania można otrzymać punkty styczności $P_{max}$ oraz $P_{min}$ oraz odpowiadające im krzywizny $c_{min}$ oraz $c_{max}$. 
	
	Ostatecznie można już dokonać przybliżenia odległości po łuku od rozpatrywanej przeszkody $p$ w następujący sposób:
	\begin{equation}\label{eq:CVM_dv_const}
	d_v(v,\omega,p)= \left\{ 
	\begin{array}{l l}
  	\min(d_c((0,0),P_{max}),d_c((0,0),P_{min})) & \quad \mbox{$c_{min}\leq c \leq c_{max}$  }\\
  	\infty & \quad \mbox{w p.p. }\\
	\end{array} \right. 
	\end{equation}

	Zaprezentowane rozumowanie prowadzi do sytuacji, w której zbiorowi przeszkód $O$ odpowiada zbiór przedziałów krzywizn o stałej odległości od przeszkody.
	W dalszej części pracy przedział będzie rozumiany jako trójka liczb postaci $([c_1,c_2],d_{1,2})$, gdzie $c_1$,$c_2$ to krzywizny okręgów wyznaczających ten przedział,
	natomiast $d_{1,2}$ jest odległością zdefiniowaną równaniem (\ref{eq:CVM_dv_const}).
	
	Ze zbioru przedziałów należy następnie utworzyć listę przedziałów $\mathbb{S}$, tak, aby zawierała
	przedziały $s$ rozłączne o odległości do najbliższej przeszkody (zgodnie z równaniem (\ref{eq:CVM_Dv})).
	W oryginalnej pracy na temat CVM \cite{CVM_2} zaproponowany został algorytm (\ref{alg:CVM_lista}) realizujący to zadanie.
	\begin{algorithm}
	\caption{tworzy listę rozłącznych przedziałów $\mathbb{S}$}
	\label{alg:CVM_lista}
	\begin{algorithmic}
	\STATE $ \mathbb{S}:= ((-\infty;\infty),L) $
	\STATE  wybierz kolejny nie dodany przedział $([c_{min};c_{max}],d)$
	\FORALL {$ s\in \mathbb{S}$}
	\IF {zbiory są rozłączne } 
	\STATE nic nie rób
	\ELSIF {s zawiera się w $([c_{min};c_{max}],d)$}
	\STATE ustaw odległość $s.d= min(s.d,d) $
	\ELSIF {s zawiera  $([c_{min};c_{max}],d)$}
		\IF{$d<s.d$}
		\STATE podziel $s$ na trzy przedziały: \\$([s.c_{min};c_{min}],s.d)$, $([c_{min};c_{max}],d)$, $([c_{max};s.c_{max}],s.d)$ 
		\ELSE
		\STATE nic nie rób
		\ENDIF
	\ELSIF {s częściowo zawiera $([c_{min};c_{max}],d)$ }
		\IF{$d<s.d$}
		\STATE podziel $s$ na dwa przedziały
			\IF{$c_{max} > s.c_{max}$}
			\STATE podziel na przedziały $([s.c_{min};c_{min}],s.d)$ $([c_{min};s.c_{max}],d)$
			\ELSE
			\STATE podziel przedziały $([s.c_{min};c_{max}],d)$ $([c_{max};s.c_{max}],s.d)$
			\ENDIF
		\ELSE
		\STATE nic nie rób
		\ENDIF
	\ENDIF
	\ENDFOR
	\end{algorithmic}
	\end{algorithm}
	\begin{figure}[H]
	\centering
	%\includegraphics[scale=0.45]{./algorytmy/CVM_segmentacja}
	\includegraphics[scale=0.30]{./algorytmy/CVM_segmentacja_new}
	\caption{ Ukazanie sensu zastosowania segmentacji} \label{fig:CVM_segmentacja}
	\end{figure}
	W zależności od rozmieszczenia przeszkód w środowisku przedziały wyznaczone przez krzywizny $c_{min}$ oraz $c_{max}$ charakteryzują się różną szerokością. Często przybliżenie zbioru długości krzywych należących do
	$[c_{min},c_{max}]$ jedną stałą wartością jest niewystarczające. Zazwyczaj przedział ten zawiera zbiór łuków których odległość od przeszkody jest istotnie mniejsza niż ta wynikająca ze wzoru (\ref{eq:CVM_dv_const}).
	Dotyczy to w szczególności przypadków, kiedy przedział $[c_{min},c_{max}]$ jest szeroki. Sytuacja taka została zilustrowana na rysunku \ref{fig:CVM_segmentacja}.
	

	Jednym z możliwych rozwiązań powyższego problemu jest dokonanie podziału przedziału $[c_{min},c_{max}]$ na kilka
	mniejszych przedziałów i do każdego z nich zastosowanie równania (\ref{eq:CVM_dv_const}) w celu wyznaczenia odległości od przeszkody. W pracy przyjęto rozwiązanie analogiczne jak opisane w \cite{majchrowski}.
	Pierwszym etapem jest wyznaczenie punktu leżącego na okręgu reprezentującym przeszkodę położonego najbliżej robota zgodnie z  metryką euklidesową. Następnie okrąg jest dzielony symetrycznie na $k$  segmentów. Postępowanie takie zostało przedstawione na rysunku \ref{fig:CVM_segmentacja2}. 
	Dla każdych dwóch sąsiadujących ze sobą punktów leżących pomiędzy punktami styczności $P_{min}$ oraz $P_{max}$
 	wyznaczane są krzywizny $c_1$ oraz $c_2$ i tworzony jest przedział. Jako odległość przyjmowana jest krótsza 
	z odległości zgodnie ze wzorem (\ref{eq:CVM_dv_const}). Tak otrzymane przedziały są dodawane do listy za pomocą algorytmu \ref{alg:CVM_lista}.
	
	\begin{figure}[H]
	\centering
	%\includegraphics[scale=0.40]{./algorytmy/CVM_segmentacja2}
	\includegraphics[scale=0.32]{./algorytmy/CVM_segmentacja2_new}
	\caption{ Efekt podziału okręgu na części} \label{fig:CVM_segmentacja2}
	\end{figure} 
\section{Algorytm RRT\\(\textit{Rapidly-Exploring Random Tree}) \label{sec:RRT:basic}}
Kolejnym zupełnie odmiennym podejściem w planowaniu bezkolizyjnej ścieżki jest losowe przeszukiwanie dopuszczalnej przestrzeni położeń robota. Szczegółowe informacje na
jego temat można znaleźć w \cite{RRT} oraz w \cite{RRT2}.
Algorytm RRT jest wydajnym algorytmem, który w czasie rzeczywistym może wyznaczyć drogę do celu, jednak nie jest ona optymalna pod względem dynamiki, kinematyki ani długości.
Podstawową strukturą danych na której operuje algorytm jest drzewo. Jako korzeń przyjmowany jest stan reprezentujący położenie robota w momencie uruchomienia algorytmu.
Budowa drzewa polega na dodawaniu kolejnych węzłów do drzewa w kierunku punktu obranego w danym kroku jako docelowy. Tymczasowy stan docelowy generowany jest w następujący sposób:
 \begin{algorithm}[H]
	
	\caption{ Funkcja obliczająca stan docelowy }
	\label{alg:chooseTarget}
	\begin{algorithmic}
	\STATE \texttt{function} \textit{chooseTarget}(\textit{goal:state}) \textit{state};
	\STATE p = \texttt{uniformRandom in} $[0.0 ...1.0]$
	\IF{$0.0<p<goalProb$} 
	  \RETURN  \texttt{goal};
	\ELSIF{ $goalProb<p<1.0$}
	  \RETURN \textit{RandomState()}
	\ENDIF
	\end{algorithmic}
  \end{algorithm}

Na listingu \ref{alg:chooseTarget} użyta została funkcja \texttt{RandomState}, zwracająca losowy punkt ze zbioru wszystkich możliwych położeń robota.
W najprostszej realizacji, może ona zwracać współrzędne punktu dobierane z rozkładem jednostajnym z określonego wcześniej obszaru.
W każdej iteracji algorytmu tymczasowy stan docelowy obliczany jest na nowo. Aktualne drzewo, na którym operuje algorytm przeszukiwane jest w celu znalezienia
węzła \texttt{nearest} położonego najbliżej tego celu. Węzeł \texttt{nearest} rozszerzany jest w kierunku tegoż celu za pomocą funkcji \texttt{extend}.
Operacja ta w najprostszym ujęciu polega na stworzeniu nowego węzła przesuniętego względem \texttt{nearest} o odcinek \texttt{rrtStep} w kierunku tymczasowego celu.
Istotne jest, aby sprawdzić czy nowo wygenerowany stan znajduje się w dopuszczalnej przestrzeni stanów \textdollar. W najprostszym wariancie można zastosować
heurystykę, sprawdzającą jedynie czy położenie to nie koliduje z żadną z przeszkód znajdujących się w danej sytuacji oraz czy robot jest w stanie do niego dotrzeć.
Bardziej zaawansowane heurystyki powinny sprawdzać czy na przykład podczas przemieszczania  się do nowej pozycji nie nastąpi kolizja z dynamiczną przeszkodą.
W ogólnym ujęciu algorytm RRT wygląda zatem następująco:
  \begin{algorithm}[H]
	\caption{ Ogólna zasada algorytmu \texttt{RRT} }
	\label{alg:RRT}
	\begin{algorithmic}
	\STATE zbuduj model eksplorowanego świata $\mapsto$ \texttt{env:environment};
	\STATE zainicjuj stan początkowy $\mapsto$ \texttt{initState:state};
	\STATE zainicjuj stan końcowy $\mapsto$ \texttt{goalState:state};
	\STATE
	\WHILE { \texttt{goalState.distance(nextState)} $>$ \texttt{minimalDistance} }
	  \STATE wybierz tymczasowy cel: \texttt{target} =  \texttt{chooseTarget(goalState)}
	  \STATE nearestState $=$ \texttt{ nearest(tree, target) }
	  \STATE \texttt{extendedState} $=$ \texttt{extend(env, nearest, target)}
	  \IF { \texttt{extendedState} $!=$ \texttt{NULL} } 
	    \STATE \texttt{addNode(tree, extended)}
	  \ENDIF
	\ENDWHILE
	\RETURN  \texttt{tree};
	\end{algorithmic}
  \end{algorithm}
W opisie algorytmu \ref{alg:RRT} użyte zostały nie omówione jeszcze funkcje, \texttt{distance} zwracająca odległość w sensie euklidesowym między dwoma punktami zawartymi w węzłach drzewa,
\texttt{addNode} dodająca węzeł do bieżącego drzewa. Parametr \texttt{rrtStep} jest natomiast wyrażony w metrach.
Tak zaimplementowana wersja algorytmu może być zastosowana jako skuteczne narzędzie do bezkolizyjnej nawigacji robotem, jednak jego wydajność można poprawić poprzez
modyfikacje opisane w kolejnym podrozdziale.

\subsection{Modyfikacje algorytmu, czyli przejście z RRT do ERRT \label{sec:RRT:extend} }
Omawiany algorytm można zmodyfikować pod kątem szybkości działania. Po pierwsze, elementy drzewa można przechowywać w strukturach ułatwiających przeszukiwanie drzewa.
W literaturze stosowane są w tym celu \texttt{KD-drzewa}. Kolejną modyfikacją jest zapamiętywanie losowo wybranych węzłów ze ścieżki znalezionej w poprzednim uruchomieniu
algorytmu. W takim wariancie tymczasowy punkt docelowy obierany jest następująco:
 \begin{algorithm}[H]
	\caption{ Zmodyfikowana funkcja obliczająca stan docelowy }
	\label{alg:chooseTargetExt}
	\begin{algorithmic}
	\STATE \texttt{function} \textit{chooseTargetExt}(\textit{goal:state}) \textit{state};
	\STATE p = \texttt{uniformRandom in}$[0.0 ...1.0]$
	\STATE i = \texttt{uniformRandom in}$[0 ...number\_of\_waypoints -1]$
	\IF{$0.0<p<goalProb$} 
	  \RETURN  \texttt{goal};
	\ELSIF{ $goalProb<p<goalProb+wayPointProb$}
	  \RETURN \textit{WayPoints[i]}
	\ELSIF{ $goalProb+wayPointProb<p<1.0$}
	  \RETURN \textit{RandomState()}
	\ENDIF
	\end{algorithmic}
  \end{algorithm}
W publikacji \cite{RRT} prawdopodobieństwo podążania do właściwego celu ostatecznie ustalane zostało na poziomie $0.1$, natomiast \texttt{wayPointProb} wynosiło $0.7$.
\begin{figure}[h]
\centering
\includegraphics[scale=0.34]{./algorytmy/tree_with_waypoints.png}
\caption{ \texttt{RRT} z zapamiętanymi węzłami z poprzedniego uruchomienia. Linią kropkowaną zaznaczono gałąź dodaną w kierunku losowego punktu,
 linią przerywaną w kierunku jednego z elementów z poprzedniej ścieżki, a linią niebieską w kierunku celu.} \label{fig:tree_with_waypoints}
\end{figure} 
Kolejną modyfikacją jest zmiana sposobu obliczania odległości, standardowo w algorytmie wykorzystywana jest odległość pomiędzy liściem drzewa a punktem docelowym. Aby uniknąć
nadmiernego rozrastania drzewa, odległość może uwzględniać dodatkowo odległość liścia od korzenia pomnożoną przez określony współczynnik wagowy.

\section{Zalety algorytmu RRT w stosunku \\do CVM}
Podczas kontynuacji prac nad rozgrywkami RoboCup zdecydowano się na zmianę algorytmu wyznaczającego bezkolizyjną ścieżkę. Zmiana została wymuszona decyzją o zmianie
modelu sterowanego robota. Bazę jezdną o napędzie różnicowym zastąpiono holonomiczną. Algorytm CVM jest algorytmem dedykowanym do robotów o napędzie różnicowym, doskonale
można ująć w nim ograniczenia na dopuszczalne prędkości. Jednak sterowanie za jego pomocą robotem holonomicznym nie pozwala na pełne wykorzystanie możliwości robota.
Dodatkowo, jak zostało udowodnione w pracy inżynierskiej \cite{inzynierka}, CVM nie sprawdza się w silnie dynamicznym środowisku. Podczas testów z przeszkodami dynamicznymi osiągał
skuteczność na poziomie $80\%$. Kolizje występowały głównie w sytuacjach, kiedy przeszkoda uderzała w robota z boku lub z tyłu. Wynikało to z faktu, że trajektoria ruchu takiej przeszkody znajdowała się poniżej osi $OY$ w układzie współrzędnych
związanym ze sterowanym robotem. Przeszkody takiego typu nie były uwzględnianie przez algorytm.
RRT pozwala na uwzględnianie przeszkód rozsianych po całej planszy. Dodatkowo umożliwia poruszanie się po bardziej złożonych trajektoriach.
Kolejną zaletą RRT jest prostota w ograniczaniu przestrzeni, z której losowane są tymczasowe punkty docelowe. Dzięki temu w prosty sposób można ograniczyć poruszanie
się robota tylko do wnętrza boiska. W przypadku CVM istniała konieczność modelowania bandy boiska jako zbioru okręgów, mnożyło to ilość obiektów z jakimi należało wykrywać
kolizje.
Kolejnym problemem CVM jest nawigacja do celu znajdującego się za robotem. W pracy inżynierskiej wprowadzono modyfikację, zmieniającą parametr odpowiedzialny za kierowanie
robota na cel, uzależniając go od orientacji robota względem celu. Intencją było wymuszenie obrotu robota w kierunku celu, w sytuacji, gdy kąt o jaki powinien obrócić się robot, aby być skierowanym
na cel ma dużą wartość.
Modyfikacja ta poprawiła skuteczność algorytmy CVM, jednak nawet dla najlepszego zestawu współczynników wagowych wynosiła ona zaledwie $80\%$. W przypadku bazy holonomicznej obracanie się w kierunku punktu docelowego nie jest
konieczne, stąd takie sterowanie nie jest optymalne.