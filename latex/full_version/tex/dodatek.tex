\appendix
\chapter[Zawartość płyty CD ]{Zawartość płyty CD}
\chaptermark{Zawartość płyty CD}	
Dołączona do pracy płyta CD zawiera następujące elementy:
\begin{itemize}
\item pliki źródłowe wykorzystanej wersji symulatora \textit{Gazebo} wraz z wprowadzonymi poprawkami opisanymi w par.~\ref{subsect:realizacjaROBOCUP} oraz zaimplementowanym
sterownikiem holonomicznej bazy jezdnej -- folder \texttt{gazebo},
\item opracowane modele robotów, oraz boiska  -- folder \texttt{modele},
\item kod źródłowy aplikacji opracowanej aplikacji (projekt w środowisku \mbox{Eclipse}\footnote{Środowisko jest dostępne pod adresem \url{www.eclipse.org}.} dla C++) -- folder \texttt{aplikacja\_sterujaca},
\item kod źródłowy  aplikacji \textit{RRT\_debug} służącej do wizualizacji działania algorytmu RRT -- folder \texttt{RRT\_debug}, 
\item otrzymane wyniki przeprowadzonych eksperymentów wraz ze skryptami do tworzenia wykresów -- folder \texttt{wyniki},
\item zrzuty ekranu prezentujące wykonane modele oraz filmy z symulatora pokazujące realizację przeprowadzonych eksperymentów -- folder \texttt{media},
\item plik \texttt{praca\_magisterska.pdf} -- wersja elektroniczna niniejszego dokumentu.
\end{itemize}


% \chapter[Instrukcja instalacji Gazebo]{Instrukcja instalacji Gazebo}
% \chaptermark{Instrukcja instalacji Gazebo}
% \todo{poprawic opis instalacji gazebo}
% \todo{zmienic numery rewizji zastosowanych aplikacji}
% Instrukcję instalacji można znaleźć w poradniku dostępnym pod adresem \url{http://playerstage.sourceforge.net/doc/Gazebo-manual-svn-html/install.html}. \newline
% W~razie problemów można skorzystać także z opisu dostępnego pod adresem \newline \url{http://www.irobotics.org/gazebo08.f8.html}. 
% Do instalacji należy użyć wersji \textit{Gazebo} dostępnej na płycie CD dołączonej do pracy 
% (jest to wersja 6782 dostępna poprzez repozytorium SVN, zawiera jednak poprawki opisane w~par.~\ref{subsect:realizacjaROBOCUP}).
%  Do poprawnej pracy \textit{Gazebo} wymagana jest obecność dodatkowych aplikacji. Najważniejsze z nich wykorzystano w niniejszej pracy w nastepujących wersjach:
% \begin{itemize}
%  \item Player (wymagany do kompilacji Gazebo) -- pobrany z repozytorium SVN w wersji 6350,
%  \item ODE pobrane z oficjalnego repozytorium SVN, wersja 1451,
%  \item OGRE w wersji 1.4.7,
%  \item pozostałe wymagane aplikacje zostały zainstalowane w wersjach zgodnych z opisem w podręczniku instalacji.
% \end{itemize}


%\input{./tex/dodatek_CVM}
\chapter[Szczegóły eksperymentów]{Szczegóły eksperymentów \label{sec:szczegoly_eksp}}
\chaptermark{Szczegóły eksperymentów}
Dodatek zawiera poglądowe rysunki przedstawiające środowiska testowe, na których przeprowadzane były eksperymenty.
Na każdym z nich zaznaczono inne roboty nie podlegające sterowaniu przez testowany algorytm oraz początkowe położenie 
i orientację sterowanego robota. 
W przypadku eksperymentów w środowisku dynamicznym zaznaczono także kierunek i zwrot prędkości z jaka poruszały się przeszkody.
Poniżej zamieszczono legendę objaśniającą znaczenie użytych symboli.
	\begin{figure}[H]
	\centering
	\includegraphics[scale=0.35]{./eksperymenty/srodowiska/legenda}
	\end{figure}	

W drugiej części dodatku zamieszczono tabelę z zastosowanymi podczas eksperymentów z użyciem algorytmu \textit{CVM} zestawami wag. Ponieważ każda ze składowych funkcji celu (\ref{eq:fcelu_cvm}) algorytmu  zwraca wartość z przedziału 
$[0;1]$ zdecydowano się na przetestowanie takich trójek liczb z tego przedziału,  które sumują się do jedności.
\section*{Środowiska testowe\label{sec:srodowiska_testowe}}
\subsection*{statyczne:}
	\begin{figure}[H]
	\centering	
	\subfloat[]{\includegraphics[scale=0.3]{./eksperymenty/srodowiska/eksperyment00}} 
	\hspace{0.5cm}
	\subfloat[]{\includegraphics[scale=0.3]{./eksperymenty/srodowiska/eksperyment01}}    \\
	\subfloat[]{\includegraphics[scale=0.3]{./eksperymenty/srodowiska/eksperyment05}} 
	\hspace{0.5cm}
	\subfloat[]{\includegraphics[scale=0.3]{./eksperymenty/srodowiska/eksperyment06}}  \\

	\end{figure}	

	\begin{figure}[H]
	\centering
	\subfloat[]{\includegraphics[scale=0.3]{./eksperymenty/srodowiska/eksperyment09}} 
	\hspace{0.5cm}
	\subfloat[]{\includegraphics[scale=0.3]{./eksperymenty/srodowiska/korytarz}}  	\\
	\subfloat[]{\includegraphics[scale=0.3]{./eksperymenty/srodowiska/prosta}} 
	\hspace{0.5cm}
	\subfloat[]{\includegraphics[scale=0.3]{./eksperymenty/srodowiska/waskie_przejscie}}  \\
	\end{figure}
	\begin{figure}[H]
	\centering
	\subfloat[]{\includegraphics[scale=0.3]{./eksperymenty/srodowiska/atak1}} 
	\hspace{0.5cm}
	\subfloat[]{\includegraphics[scale=0.3]{./eksperymenty/srodowiska/atak2}}
	\end{figure}
\subsection*{dynamiczne:}
	\begin{figure}[H]
	\centering
	\subfloat[]{\includegraphics[scale=0.3]{./eksperymenty/srodowiska/dynamiczne/eks01}} 
	\hspace{0.5cm}
	\subfloat[]{\includegraphics[scale=0.3]{./eksperymenty/srodowiska/dynamiczne/eks02}}    \\
	\end{figure}
	\begin{figure}[H]
	\centering
	\subfloat[]{\includegraphics[scale=0.3]{./eksperymenty/srodowiska/dynamiczne/eks03}} 
	\hspace{0.5cm}
	\subfloat[]{\includegraphics[scale=0.3]{./eksperymenty/srodowiska/dynamiczne/eks04}}  \\
	\subfloat[]{\includegraphics[scale=0.3]{./eksperymenty/srodowiska/dynamiczne/eks05}} 
	\hspace{0.5cm}
	\subfloat[]{\includegraphics[scale=0.3]{./eksperymenty/srodowiska/dynamiczne/eks06}}  	\\
	\end{figure}	
	\begin{figure}[H]
	\centering
	\subfloat[]{\includegraphics[scale=0.3]{./eksperymenty/srodowiska/dynamiczne/eks07}} 
	\hspace{0.5cm}
	\subfloat[]{\includegraphics[scale=0.3]{./eksperymenty/srodowiska/dynamiczne/eks08}}  \\
	\subfloat[]{\includegraphics[scale=0.3]{./eksperymenty/srodowiska/dynamiczne/eks09}} 
	\hspace{0.5cm}
	\subfloat[]{\includegraphics[scale=0.3]{./eksperymenty/srodowiska/dynamiczne/eks10}}
	\end{figure}	

\section*{Zestawy wag używane w eksperymentach \label{sec:zestawy_wag}}
\begin{table}[H]
\centering
\scalebox{0.8}{
\begin{tabular}[t]{|c ||c |c| }
\hline
Zestaw wag & $goalProb$ & $wayPointProb$ \\ \hline\hline
1 & 0.0 & 0.0\\ \hline
2 & 0.0 & 0.2\\ \hline
3 & 0.0 & 0.3\\ \hline
4 & 0.0 & 0.4\\ \hline
5 & 0.0 & 0.5\\ \hline
6 & 0.0 & 0.6\\ \hline
7 & 0.0 & 0.7\\ \hline
8 & 0.0 & 0.8\\ \hline
9 & 0.0 & 0.9\\ \hline
10 & 0.0 & 1.0\\ \hline
11 & 0.1 & 0.0\\ \hline
12 & 0.1 & 0.1\\ \hline
13 & 0.1 & 0.2\\ \hline
14 & 0.1 & 0.3\\ \hline
15 & 0.1 & 0.4\\ \hline
16 & 0.1 & 0.5\\ \hline
17 & 0.1 & 0.6\\ \hline
18 & 0.1 & 0.7\\ \hline
19 & 0.1 & 0.8\\ \hline
20 & 0.1 & 0.9\\ \hline
21 & 0.2 & 0.0\\ \hline
22 & 0.2 & 0.1\\ \hline
23 & 0.2 & 0.2\\ \hline
24 & 0.2 & 0.3\\ \hline
25 & 0.2 & 0.4\\ \hline
26 & 0.2 & 0.5\\ \hline
27 & 0.2 & 0.6\\ \hline
28 & 0.2 & 0.7\\ \hline
29 & 0.2 & 0.8 \\ \hline
30 & 0.3 & 0.0 \\ \hline
31 & 0.3 & 0.1 \\ \hline
32 & 0.3 & 0.2 \\ \hline
33 & 0.3 & 0.3 \\ \hline
\end{tabular}
}
\hspace{0.5cm}
\scalebox{0.8}{
\begin{tabular}[t]{|c ||c |c| }
\hline
Zestaw wag &$goalProb$ & $wayPointProb$\\ \hline\hline
34 & 0.3 & 0.4 \\ \hline
35 & 0.3 & 0.5 \\ \hline
36 & 0.3 & 0.6 \\ \hline
37 & 0.3 & 0.7 \\ \hline
38 & 0.4 & 0.0 \\ \hline
39 & 0.4 & 0.1 \\ \hline
40 & 0.4 & 0.2 \\ \hline
41 & 0.4 & 0.3 \\ \hline
42 & 0.4 & 0.4 \\ \hline
43 & 0.4 & 0.5 \\ \hline
44 & 0.4 & 0.6 \\ \hline
45 & 0.5 & 0.0 \\ \hline
46 & 0.5 & 0.1 \\ \hline
47 & 0.5 & 0.2 \\ \hline
48 & 0.5 & 0.3 \\ \hline
49 & 0.5 & 0.4 \\ \hline
50 & 0.5 & 0.5 \\ \hline
51 & 0.6 & 0.1 \\ \hline
52 & 0.6 & 0.2 \\ \hline
53 & 0.6 & 0.3 \\ \hline
54 & 0.6 & 0.4 \\ \hline
55 & 0.7 & 0.0 \\ \hline
56 & 0.7 & 0.1 \\ \hline
57 & 0.7 & 0.2 \\ \hline
58 & 0.7 & 0.3 \\ \hline
59 & 0.8 & 0.0 \\ \hline
60 & 0.8 & 0.1 \\ \hline
61 & 0.8 & 0.2 \\ \hline
62 & 0.8 & 0.2 \\ \hline
63 & 0.9 & 0.0 \\ \hline
64 & 0.9 & 0.1 \\ \hline
65 & 1.0 & 0.0 \\ \hline
\end{tabular}
}
\end{table}

