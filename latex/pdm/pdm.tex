\documentclass[11pt,onecolumn,a4paper,final]{article}

\usepackage[left=2.5cm,top=2.5cm,right=2.5cm,bottom=1.5cm,includehead,includefoot]{geometry}

%\usepackage[noend]{algorithmic}
%\usepackage{algorithm}
\usepackage[MeX]{polski}
\usepackage{url}
\usepackage{graphicx}
\usepackage{float}
\usepackage[justification=centering]{caption}
%\usepackage{subfloat}
\usepackage{subfig}
\usepackage{amsmath}
\usepackage{amssymb}
%\usepackage[countmax]{subfloat}
\usepackage{wrapfig}


\usepackage{color}
\linespread{1.3}


% kodowanie: latin2, utf8 lub cp1250

\usepackage[utf8]{inputenc}
%\floatname{algorithm}{Algorytm}


%\usepackage{fancyhdr}
%\setlength{\headheight}{15pt}
%\pagestyle{fancy}
%\lhead{\fancyplain{}{\nouppercase\leftmark}}
%\chead{}
%\rhead{}
%\lfoot{}
%\cfoot{\fancyplain{}{\thepage}}
%\rfoot{}


\begin{document}
\begin{minipage}{0.7\textwidth}
 
\begin{flushleft} \large
Maciej Gąbka \\
nr indeksu:198404\\
\texttt{M.Gabka@stud.elka.pw.edu.pl}
\end{flushleft}

\begin{flushright}
\date{Warszawa, dn. 09.02.2009}
\end{flushright}
\end{minipage}

\title{\bfseries Sprawozdanie z pracowni dyplomowej magisterskiej(PDM-1)}
\maketitle
Prace wykonane w ramach pracowni w minionym semestrze:
\begin{enumerate}
 \item  Dokonano przeglądu dostępnej literatury dotyczącej rozwiązań stosowanych przez najlep-
sze drużyny biorące udział w rozgrywkach \emph{Ligi RoboCup}. Szczególną uwagę zwrócono na
ligę Small Size, która będzie stanowić środowisko testowe opracowywanego algorytmu.
Jednak ze względu na to, iż stosowane są w niej głównie centralne sposoby sterowania
drużyną, dokonano także przeglądu rozwiązań wdrażanych w rozgrywkach Middle Size,
gdzie zdecentralizowane sterowanie jest stosowane powszechnie.
\item Zaprojektowano schemat i elementy funkcjonalne aplikacji mającej realizować algorytm
sterowania agentem w meczu Ligi RoboCup.
\item Przeprojektowano model robota,  opracowany w ramach  pracy inżynierskiej. Zmieniono bazę jezdną robota z różnicowej na omnikierunkową oraz zaimplementowano sterownik do takiego modelu.
w rozgrywce \emph{Small Size}

\end{enumerate}

Plan pracy magisterskiej na kolejne semetry:
\begin{enumerate}
 \item implementacja klas odpowiadających za sterowanie agentem i realizowanie strategii gry,
 \item konfrontacja stworzonego rozwiązania z drużyną agentów sterowaną w sposób centralny.
\end{enumerate}

\end{document}