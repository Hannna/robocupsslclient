\chapter{Algorytmy unikania kolizji}
\begin{abstract}
 W rozdziale zostanie szerzej zaprezentowany jeden z algorytmów unikania kolizji jakim jest RRT(\texttt{Rapidly-Exploring Random Tree}).
Poruszona zostanie kwestia powodów, dla których wybrano tę, a nie inną metodę. Uzasadnione zostenie odejscie od algorytmu opracowanego w ramach pracy 
inżynierskiej(CVM \texttt{Curvature Velocity Method}). Ponadto opisane zostaną inne metody planowania ścieżki, omówione zostane ich właściwości. Na tej podsatwie zostanie 
uzasadniony wybór algorytmu RRT. Omówione zostaną także szczegóły implementacji algorytmu.
\end{abstract}

	\section{Krótki przegląd algorytmów unikania kolizji}
	Krótki przegląd metod unikania kolizji  z odsyłaczami do literatury szerzej opisującej problem. 
	Wstępne odrzucenie globalnych metod planowania ruchu, ze względu na dużą złożoność obliczeniową,
	a co z tym idzie  niską efektywność w dynamicznym środowisku.
	
	\begin{itemize}
	\item algorytm BUG
 	\item algorytm VFH
	\item technika dynamicznego okna	
	\item algorytm CVM
	\item metoda pól potencjałowych
	\item algorytm RRT
	\end{itemize}
	
	\section{Zalety algorytmy RRT w stosunku \\do CVM}
	\begin{itemize}
	 \item Pokazanie zalet algorytmu oraz prostoty jego wykorzystania w przypadku robotów o napędzie holonomicznym.
	 \item Szybka możliwość sterownania czasem trwania algorytmu, poprzez sterowanie liczbą węzłow drzewa.
	 \item Proste modyfikacje algorytmu uwzględniające ograniczenia wynikające z dynamiki oraz kinematyki robota.
	\end{itemize}
	
	\section{Szczegółowa zasada algorytmu RRT}
	\begin{itemize}
		\item opis algorytmu budującego drzewo
		\item opis sposobu określania przestrzeni osiągalnych stanów robota w zależnośći rozpatrywanego węzła w drzewie
		\begin{itemize}
			\item uwzglednianie kolizji na drodze od położenia bierzącego punktu docelowego	
			\item przeszkody opisywane za pomocą okręgów( promień okręgu przeszkody podczas planowania ścieżki jest
			wiekszy niz podczas wykrywania kolizji)
		\end{itemize}

		\item modyfikacje wprowadzone do podstawowej wersji algorytmu
		\begin{itemize}
		\item \texttt{waypoints}
		\item ograniczenie na maksymalna liczbę wezłów
		\item uwzględnienie prędkośći przeszkód
		\end{itemize}
	\end{itemize}


	\section{Szczegóły implementacji zastosowanego \\algorytmu }
	\begin{itemize}
	 \item opis wykorzystywanych struktur danych
	 \item omówienienie funkcji losującej stan docelowy
	\end{itemize}
	
	\section{Wyznaczanie prędkości prowadzącej do punktu docelowego}
	Wyznaczeniee punktu znajdującego się na bezkolizyjnej ścieżcze prowadzącej do celu nie rozwiązuje problemu nawigacji robota, kolejnym krokiem
	jest wyznaczenie prędkości prowadzących robota do celu.
	W paragrafie zostanie opisany mechanizm zastosowany do wyznaczania prędkości robota w dwóch przypadkach:
	\begin{itemize}
	 \item dojazd do celu,
	 \item poruszanie się do celu z piłką.
	\end{itemize}