\chapter{Projekt Player/Stage/Gazebo}
	Rozdział powinien zawierać krótki opis symulatora. Najwięcej uwagi należy poświecic konstrukcji modeli robotów, oraz wprowadzonym
	poprawkom do symulatora, tj. tarciu w dwóch płaszczyznach oraz tarciu przy ruchu obrotowym.
	\section{Koncepcja projektu}
	Krótkie wprowadzenie - symulator open source, stworzony w celu modelowania
	robotów, umożliwiający symulowanie ich działania, ale także dostarczający interfejsy pozwalające na
	sterowanie rzeczywistymi robotami. (Licencja GPL, podać strony www oraz wiki).
	\section{Architektura}
	Schemat architektury systemu i przepływu informacji:
 	Gazebo + Player + Stage + aplikacja kliencka + rzeczywisty robot.
 	Krótki opis każdego z komponentów i jego przydatności w kontekście rozważanego zadania (symulacja ligi).
 	\subsection{Stage}	
 	\subsection{Gazebo}
 	\subsection{Player}
 	
	\section{Modelowanie obiektów}
	Krótki wstęp na temat modelowania, opisać bibliotekę stosowana do wizualizacji, do wykrywania kolizji.

	\section{Zasady modelowania w Gazebo 0.10}
	Ta część rozdziału będzie opisywać ogólne zasady modelowania w symulatorze Gazebo w wersji 0.810. Zostanie
	pokazany schemat (krok po kroku) tworzenia prostego świata (plik .world), obiektów, połączeń między nimi, 
	sposobu kontroli przez aplikację kliencką. Proste przykłady, mające dać ogólne pojęcie o możliwościach środowiska,
	bez wchodzenia w szczegóły.
	
	\section{Realizacja środowiska ligi Robocup}
	Prezentacja modeli wykonanych na potrzeby ligi: boisko oraz roboty, wspomnieć o konieczności napisania własnego 
	kontrolera. Screeny prezentujące modele. Pełne kody źródłowe w xml opisujące modele.