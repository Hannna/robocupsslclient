\documentclass[11pt,onecolumn,a4paper,final]{article}

\usepackage[left=2.5cm,top=2.5cm,right=2.5cm,bottom=1.5cm,includehead,includefoot]{geometry}
\usepackage[MeX]{polski}
\usepackage{url}
\usepackage{graphicx}
\usepackage{float}
\usepackage[justification=centering]{caption}
\usepackage{subfig}
\usepackage{amsmath}
\usepackage{amssymb}
\usepackage{wrapfig}
\usepackage{color}
\usepackage[utf8]{inputenc}

\linespread{1.3}

\begin{document}
 

\begin{flushright}
	Warszawa, dn. 24.01.2010
\end{flushright}

\vspace{-1.45cm}

\begin{flushleft}
	Maciej Gąbka \\  nr indeksu: 198404 \\ \url{M.Gabka@stud.elka.pw.edu.pl}
\end{flushleft}

\vspace{0.5cm}

\begin{center}
	\LARGE \bfseries Sprawozdanie \\ z przygotowania pracy dyplomowej magisterskiej (PDM-2)
\end{center}

\vspace{1cm}

\begin{center}
\textbf{Temat pracy}: Zcentralizowany algorytm sterowania drużyną robotów w rozgrywkach RoboCup\\
\textbf{Promotor}: dr hab. in. Jarosław Arabas\
\end{center}

\section*{Opis algorytmu sterujacego drużyną}
Algorytm sterujący drużyną robotów zostanie zaimplementowany w oparciu o architekturę \mbox{\texttt{STP - Skill Tactics Play}} szerzej opisaną 
w \cite{stp}. Podejście to zakłada istnienie następujących modułów:
\begin{enumerate}
  \item moduł zawierający informacje o świecie,
  \item moduł umożliwiający ocenę sytuacji na planszy (pozwalający na ocenę atrakcyjności zachowań, punktów docelowych etc),
  \item moduł sztucznej inteligencji,
  \item moduł nawigacji robota; 
  moduł ten powinien być odpowiedzialny za tworzenie bezkolizyjnej ścieżki prowadzącej do zadanego celu,
  \item moduł sterowania ruchem robota;
  moduł ten powinien wyznaczyć optymalne prędkości prowadzące do zadanego punktu (problem bezwładności robota,
  wyhamowanie przed punktem docelowym etc.),
  \item moduł bezpośrednio odpowiedzialny za sterowanie warstwą fizyczną robota (zadawanie prędkości liniowej kątowej, uruchamianie urządzenia do prowadzenia 
  piłki(\textit{dribbler-a}), kopnięcie piłki).
\end{enumerate}
 
Nazwa \texttt{STP} odnosi się do sposobu realizacji modułu sztucznej inteligencji.
\subsection*{Podstawowe pojęcia}
Architektura \texttt{STP} zakłada planowanie działań drużyny na 3 poziomach.
\begin{enumerate}
  \item Poziomem najwyżej w hierachii jest \texttt{Play}. Przez to pojecie rozumiany jest plan gry dla całej drużyny, uwzględniana jest tutaj koordynacja
  poczynań pomiędzy zawodnikami. Plan zakłada przydział roli dla każdego zawodnika. W obrębie danego planu zawodnik wykonuje swoją rolę, aż do momentu
  zakończenia danego planu lub wyznaczenie kolejnego.
  \item Przez \texttt{Tactics} rozumiany jest plan działań dla jednej roli. Można zatem to rozumieć jako plan działań na szczeblu robota prowadzący
 do osiągnięcia efektu. Przykładem może być strzał na bramkę. Robot dostaje polecenie oddania strzału na bramkę, zatem plan jego poczynań ma doprowadzić
 do sytuacji, w której osiągnie on pozycję umożliwiającą strzał na bramkę z zadanym powodzeniem.
 Plan na szczeblu pojedynczego robota jest wykonywany do momentu zmiany planu gry calej druzyny.
 Przykładowe plany działań dla robotów:
 \begin{itemize}
  \item strzał na bramkę,
  \item podanie piłki,
  \item odebranie podania,
  \item blokowanie innego robota,
  \item wyjście na pozycję,
  \item bronienie pozycji,
  \item dryblowanie z piłką.
 \end{itemize}

  \item Pojęcie \texttt{Skills} odnosi się do konkretnych umiejętności robota, takich jak:
    \begin{itemize}
    \item doprowadzenie piłki do celu piłki,
    \item przemieszczenie robota do celu,
    \item podążanie za innym robotem,
    \end{itemize}
  Na tym szczeblu zachowanie robota zmieniane jest w każdym kroku gry. Z każdego zadania, w każdym kroku gry określone musi być przejście
  albo do nowego zadania albo wykonywanie nadal tego samego zadania. Przykładowo jeśli zlecimy robotowi przemieszczenie do celu z piłką i piłka 
  odskoczy robotowi, to powinien do niej podjechać i ponownie prowadzić lub jeśli nastąpi dobra okazja do strzału wykorzystać ją.
 
\end{enumerate}
Koordynacja poczynań drużyny zapewnione jest na najwyższym poziomie. Dodatkowo każdy poziomo wprowadza dodatkowe parametry wykorzystywane przy wykonywaniu
zadania na najniższym poziomie.
Wykonywanie każdego planu \texttt{Play} wymaga spełnienia określonych predykatów, np. czy mamy rozpoczęcie gry z autu, czy wystąpił rzut rożny.
\section*{Prace wykonane w tym semestrze}
\begin{enumerate}
 \item Dostosowanie symulatora \textbf{Gazebo} \footnote{\url{http://playerstage.sourceforge.net/}} do potrzeb pracy magisterskiej:
    \begin{enumerate}
    \item usunięcie błędów pamięci,
    \item implementacja tarcia dla obiektów, którym prędkość nie jest zadawana w każdym kroku symulacji (np. piłka podczas rozgrywki),
    \item rozszerzenie modelu dynamiki niektórych obiektów o tarcie w kierunku prostopadłym do kierunku ruchu obiektu (np. koła robota holonomicznego
    charakteryzują się brakiem tarcia w kierunku prostopadłym do osi ich obrotu),
    \item poprawienie modelu robota z omnikierunkową bazą jezdną  stworzonego w poprzednim semestrze.
    \end{enumerate}

    \item Implementacja algorytmu \texttt{Rapidly-Exploring Random Trees \cite{errt}}:
    \begin{enumerate}
     \item przetestowanie algorytmu pod względem omijania przeszkód,
     \item implementacja serializacji ścieżki dla poszczególnych robotów do pliku xml,
    \end{enumerate}
  
    \item Stworzenie aplikacji umożliwiającej krokowe śledzenie zachowań robota:
    \begin{enumerate}
    \item aplikacja umożliwia wczytywanie pliku xml z zapisem ścieżki robota,
    \item podgląd zbudowanego drzewa \textbf{RRT} w kolejnych krokach gry,
    \item podgląd planu gry dla danego robota.
    \end{enumerate}

    \item Implementacja punktów 1,4 oraz 6 wymienionych przy opisie algorytmu sterującego druzyną.
\end{enumerate}

\begin{thebibliography}{9}
 \bibitem{stp}
	B.Browning, J.Bruce, M.Bowling, M.Veloso:
	\emph{STP: Skills, tactics and plays for multi-robot control
             in adversarial environments.}
	  Carnegie Mellon University, 2004.

\bibitem{errt}
	J.Bruce, M.Veloso: 
	\emph{Real-Time Randomized Path Planning for Robot Navigation.}
	  Carnegie Mellon University.

\bibitem{wmasewicz}
	B.Browning, J.Bruce, M.Bowling, M.Veloso:
	\emph{Multi-Robot Team Response to a Multi-Robot
             Opponent Team.}
	  Carnegie Mellon University.
\end{thebibliography}

\end{document}
