\chapter{Testy algorytmu nawigacji robota}
W rozdziale zostaną zaprezentowane testy warstwy odpowiadającej za nawigację robotem, a więc sprawdzony zostanie algorytm RRT oraz
algorytm wyznaczania prędkości prowadzącej do zadanego punktu. Testy zostana przeprowadzone zarówno w środowisku statycznym jak i dynamicznym.
\section{Zachowanie algorytmu RRT w środowisku statycznym}
W paragrafie zostaną opisane eksperymenty dowodzące poprawność działania algorytmu RRT. Ponadto eksperyment ma pomóc w dobraniu odpowiednej wagi parametru p odpowiadającego za kierowanie robota do punktu docelowego.\\
Do wizualizacji ścieżki utworzonej przez robota w każdym kroku algorytmu RRT zostanie użyta osobna aplikacja napisana w języku JAVA.
\textbf{Plan eksperymentów:}
\begin{itemize}
 \item w jednej linii ustawionych kilka robotów  a za nimi piłka, jeden z robotów ma za zadanie dotrzeć do piłki,
 \item przedmiotem badań będzie czas wykonania algorytmu na który bezpośrednio przekłada się maksymalna dopuszczalna liczba węzłów drzewa oraz parametr p określający prawdopodobieństwo, że w kolejnym kroku algorytmu robot będzie
 poruszał się do celu,
\end{itemize}

Na podstawie eksperymentów zostanią sporządzone rysunki wyznaczające trajektorie po jakiej poruszał się robot na drodze do celu oraz średni czas dojazdu dla wybranych parametrów.

\section{Środowisko dynamiczne}
W paragrafie zostaną opisane eksperymenty mające na celu sprawdzenie efektywności algorytmu RRT w środowisku
zmiennym w czasie. Sprawdzona zostanie efektywność modyfikacji algorytmu RRT pod kątem pracy w takim środowisku (m.in. predykcja położeni ruchomych przeszkód),
\textbf{Plan eksperymentów:}
\begin{enumerate}
 \item pierwsze eksperymenty mają na celu wykazanie poprawności działania algorytmu RRT w środowisku dynamicznym,
 a więc za pomocą algorytmu sterowany będzie jeden robot, reszta natomiast poruszać się bedzie po stałych trajektoriach ze stałą prędkośćią,
 \item wszyscy zawodnicy obu drużyn sterowani za pomocą algorytmu RRT kierowani są do losowych punktów docelowych,
 przedmiotem eksperymentu jest liczba wszystkich kolizji w przeciągu całego czasu trwania eksperymentu.
\end{enumerate}

\section{Prowadzenie piłki}
W paragrafie zostaną opisane testy sprawdzające poruszanie się robota w kierunku zadanego punktu docelowego w sytuacji kiedy prowadzi on piłkę.
\begin{enumerate}
 \item opisanie dodatkowych ograniczeń wynikających z prowadzenia piłki,
 \item porównanie z wynikami otrzymanymi w porzednich paragrafach.
\end{enumerate}

