\chapter{Architektura aplikacji sterującej drużyną robotów}
W rozdziale zostanie zaprezentowana architektura \mbox{\texttt{STP - Skill Tactics Play}} szerzej opisana 
w \cite{stp}. Podejście to zakłada istnienie następujących modułów:
\begin{enumerate}
  \item moduł zawierający informacje o świecie,
  \item moduł umożliwiający ocenę sytuacji na planszy (pozwalający na ocenę atrakcyjności zachowań, punktów docelowych etc),
  \item moduł sztucznej inteligencji,
  \item moduł nawigacji robota; 
  moduł ten powinien być odpowiedzialny za tworzenie bezkolizyjnej ścieżki prowadzącej do zadanego celu,
  \item moduł sterowania ruchem robota;
  moduł ten powinien wyznaczyć optymalne prędkości prowadzące do zadanego punktu (problem bezwładności robota,
  wyhamowanie przed punktem docelowym etc.),
  \item moduł bezpośrednio odpowiedzialny za sterowanie warstwą fizyczną robota (zadawanie prędkości liniowej kątowej, uruchamianie urządzenia do prowadzenia 
  piłki(\textit{dribbler-a}), kopnięcie piłki).
\end{enumerate}

Zostanie opisana rola każdego z modułów, ze szczególnym uwzględnieniem modułu sztucznej inteligencji.
Architektura \texttt{STP} zakłada planowanie działań drużyny na 3 poziomach.
\begin{enumerate}
 \item Poziom położony najwyżej w hierachii: \texttt{Play}.
 \item \texttt{Tactics}
 \item \texttt{Skills} 
     \begin{itemize}
    \item doprowadzenie piłki do celu piłki,
    \item przemieszczenie robota do celu,
    \item podążanie za innym robotem,
    \end{itemize}
\end{enumerate}


