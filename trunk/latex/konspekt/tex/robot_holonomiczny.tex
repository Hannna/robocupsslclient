\chapter{Sterowanie modelem robota z \emph{Small-size League} }

\begin{abstract}
W rozdziale zostanie zaprezentowana trójkołowa holonomiczna baza jezdna. Wyjaśniony zostanie sposób w jaki osiągnieto omnikierunkowość robota.
Dodatkowo zostanie omówiony algorytm wyznaczania prędkośći kół robota przy zadanej prędkości liniowej i kątowej robota.
Zaprezentowany zostanie także problem poruszania się robota z piłką i ograniczenia z niego wynikające.
\end{abstract}


	\section{Omówienie omnikierunkowej bazy jezdnej}
		\subsection{Opis położenia kół}
		\subsection{Opis kinematyki oraz dynamiki bazy}


	\section{Dryblowanie z piłką}
	\begin{itemize}
	 \item ograniczenia wynikające z budowy \texttt{dribblera},
	 \item wyznaczanie dopuszczalnych prędkośći.
	\end{itemize}

