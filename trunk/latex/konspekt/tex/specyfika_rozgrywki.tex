\chapter{Specyfika rozgrywki Robocup}
\begin{abstract}
W rozdziale zostaną zawarte teoretyczne rozważania na temat \\przebiegu pojedynczej 
rozgrywki drużyny robotów.
Umotywowana zostanie konieczność planowanie ruchu na dwóch poziomach:
	\begin{itemize}
	\item globalnym-strateg
	\item lokalnym-CVM 
	\end{itemize}
Ponadto zaprezentowana zostanie koncepcja robota jako agenta, oraz podejście do pisania algorytmów sterujących takim 
robotem  bazujące na stanach.
\end{abstract}

	\section{Zawodnik jako agent }
	\begin{itemize}
	 \item Zastosowanie podejścia agentowego w robotyce
		\begin{itemize}
		 \item rodzaje agentów
		 \item zawodnik ligi F180 jako ślepy agent
		\end{itemize}
	 \item Sterowanie agentami
		\begin{itemize}
		 \item opis za pomocą  stanów i funkcji przejść
		 \item zachowanie jako ciąg stanów
		 \item przykłady zachowań podczas rozgrywki
		\end{itemize}

	\end{itemize}

	\section{Opis rozwiązania docelowego}
	W paragrafie zostanie przedstawiony zarys rozwiązania docelowego, a więc  zaprezentowana koncepcja algorytmu
	do sterowania grupą robotów, składająca się z dwóch warstw: lokalnej , czyli sterowanie w obrębie jednego
 	agenta, polegające jedynie na unikaniu kolizji, oraz globalnej czyli uwzględniającej strategie gry oraz
	synchronizację pomiędzy agentami.
		\subsection{Unikanie kolizji}
		\begin{itemize}
		 \item  Rola algorytmu unikania kolizji
		 \item  Pożądana funkcjonalność
			\begin{itemize}
			 \item zwrotność robota
			 \item szybkość reakcji
			 \item dojazd do celu pod odpowiednim kątem
			\end{itemize}
		\end{itemize}

		\subsection{Planowanie strategii gry}
		\begin{itemize}
		\item  Opis budowania strategii oparty na planach
		\item  Opis trywialnej strategii zastosowanej w eksperymencie
		\end{itemize}
