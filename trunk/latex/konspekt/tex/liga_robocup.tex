\chapter{Liga Robocup}

\begin{abstract}
 Rozdział zostanie w całości poświęcony opisowi projektu Robocup.
Więcej informacji na ten temat należy poszukiwać na oficjalnej stronie projektu \url{http://www.robocup.org}
Zaprezentowana zostanie pokrótce historia samej idei projektu, jej rozwój oraz  poszczególne ligi wchodzące w skład całego projektu.
Głównie zostanie omówiona \emph{Small-size League}, ze szczególnym uwzględnieniem modelu robote i ograniczeń z tym związanych.
\end{abstract}


	\section{Omówienie projektu robocup}
		\subsection{Krótki opis genezy}
		\subsection{Podział na ligi robotów}
		\begin{itemize}
			\item Najstarsza z lig, liga symulacyjna
			\item Omówienie ligi \emph{Small-size League}
			\item Omówienie ligi \emph{Middle-size League}
			\item Omówienie ligi \emph{Standard Platform League }
			\item Omówienie ligi \emph{Humanoid League}
			\item \emph{RoboCupRescue} - wykorzystanie dorobku robotyki w służbie ludziom
		\end{itemize}


	\section{Specyfika ligi small-size (F180)}
	W paragrafie zostaną ujęte i dokładnie opisane zasady ligi(dzieki temu można będzie wyjaśnić rodzaj architektury zastosowanej w
	aplikacji sterującej). Po krótce zostanie przedstawiony model robota wykorzystywany w lidze ze szczególnym uwzględnieniem 
	elementów służących do podawania oraz prowadzenia piłki. Dodatkowo zostanie zaprezentowany schemat komunikacji wykorzystywany w lidze.
	\subsection{Zasady}
		\begin{itemize}
		 \item zaprezentowanie wymogów technicznych dotyczących :
			\begin{itemize}
 			\item{wymiarów robota}
			\item{wymiarów boiska}
			\item{liczby robotów}
			\item{sposobów prowadzenia piłki}  
			\end{itemize}
 		\item omówienie podstawowych zasad takich jak:
			\begin{itemize}
			 \item sygnalizowanie przewinień
			 \item rola arbitra podczas rozgrywki
			\end{itemize}
		\end{itemize}


	\subsection{Budowa robota}
		\begin{itemize}
	 	\item Zaprezentowanie konstrukcji robotów z ligi \emph{small-size}
			\begin{itemize}
			\item omówienie bazy jezdnej
			\item zaprezentowanie urządzenia umożliwiającego prowadzenie piłki
 			\item rola znaczników umieszczonych na konstrukcji mechanicznej robota
			\item dostępne czujniki (określone przez regulamin)
			\end{itemize}
		\end{itemize}
