\begin{picture}(0,110)

\put(-15,90){\textbf{Oznaczenia}}

\color{orange}
\put(0,70){\circle*{4.00}}
\color{black}
\put(10,67){Pozycja piłki (docelowa)}


\color{blue}
\put(0,50){\circle{14.14}}
\color{black}
\put(10,47){Pozycja początkowa robota}

\color{black}
\put(0,30){\circle{14.14}}
\put(10,27){Pozycja początkowa przeszkody}


\color{blue}
\put(180,50){\circle*{1.2}}
\put(182,50){\circle*{1.2}}
\put(184,50){\circle*{1.2}}
\put(186,50){\circle*{1.2}}
\put(188,50){\circle*{1.2}}
\put(190,50){\circle*{1.2}}
\put(194,50){\circle*{5}}


\color{black}
\put(180,30){\circle*{1.2}}
\put(182,30){\circle*{1.2}}
\put(184,30){\circle*{1.2}}
\put(186,30){\circle*{1.2}}
\put(188,30){\circle*{1.2}}
\put(190,30){\circle*{1.2}}
\put(194,30){\circle*{5}}

\color{black}
\put(170,65){Trajektoria i pozycja końcowa:}
\put(200,47){robota}
\put(200,27){przeszkody}

\put(-15,0){\textbf{Uwaga}}
\put(-15,-20){Im bardziej oddalone są od siebie punkty oznaczające trajektorię,}
\put(-15,-35){tym większa jest prędkość obiektu}
\end{picture}

