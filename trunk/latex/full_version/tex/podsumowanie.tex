\chapter[Podsumowanie]{Podsumowanie}
\chaptermark{Podsumowanie}
Podczas realizacji niniejszej pracy stworzono środowisko symulacyjne umożliwiające modelowanie rozgrywek robotów z ligi \emph{Small-Size League}. Ponieważ wcześniejsze prace, opisane w~\cite{inzynierka}
prowadzono na platformie Player/Stage/Gazebo, zdecydowano się na dalszą pracę z tym środowiskiem. Uaktualniono natomiast wersję stosowanego oprogramowania. Opracowano także nowe modele robotów
oraz boiska. Wykorzystywane w trakcie eksperymentów modele robotów wzorowane były na rzeczywistych zawodnikach rozgrywek RoboCup. Zmiana bazy jezdnej na wielokierunkową wymusiła implementację nowego sterownika
robota. Przygotowane środowisko, może posłużyć do dalszych prac na rozgrywkami robotów w piłkę nożną.

W ramach pracy zaimplementowano także algorytm unikania kolizji RRT. Zachowanie algorytmu sprawdzone zostało w środowiskach testowych wykorzystywanych w~\cite{inzynierka}. Dzięki takiemu podejściu
można było dokonać porównania RRT z wcześniej stosowanym CVM. W wyniku przeprowadzonych eksperymentów udowodniono, że w danych sytuacjach algorytm RRT osiąga dużo lepsze wyniki niż CVM.
Wybrana metoda rozwiązała także kilka problemów, z którymi CVM nie był w stanie sobie poradzić. Poprzednie rozwiązanie, osiągało dużo słabsze efekty w sytuacjach, kiedy cel znajdował się za robotem.
Algorytm był także mało skuteczny, gdy punkt docelowy znajdował się za zaporą utworzoną z kilku robotów. W przypadku RRT nie zauważono takich zachowań.
Sama metoda okazała się też dużo mniej wrażliwa na dobór parametrów z jakimi jest ona uruchamiana (zupełnie przeciwnie zachowywał się CVM). W przypadku silnie dynamicznego środowiska
jest to bardzo pożądana cecha. Zastosowany algorytm okazał się także szybką metodą, co ma duże znaczenie w rozgrywkach.

W dalszej części został omówiony i zaimplementowany algorytm planowania i koordynacji działań robotów wzorowany na architekturze STP opisany w \cite{stp}. W ramach pracy zrealizowano w pełni dwie
warstwy z oryginalnego rozwiązania (\texttt{skills} i \texttt{tactics}). Stworzone oporogramowanie umożliwa wykonywanie robotom podstawowych elementów gry w piłkę nożna (prowadzenie piłki, strzał na bramkę,
podanie piłki do innego zawodnika), jak i bardziej złożonych zachowań. Algorytm został przetestowany podczas realizacji wybranych zadań eliminacyjnych,
którym musieli sprostać uczestnicy rozgrywek RoboCup w ostatnich latach. W tym celu stworzono także prostą wartswę \texttt{play}, zapewniającą koordynację w obrębie drużyny
oraz opracowano plany gry rozwiązujące zadania testowe. Otrzymane wyniki zostały porównane z rezultatami uczestników mistrzostw. W przypadku pierwszego eksperymentu zaimplementowane
rozwiązanie plasowałoby się na trzecim miejscu, natomiast w przypadku drugiego eksperymentu realne byłoby miejsce drugie. Należy jednak oczywiście pamietać, że sterowanie symulowanym
robotem jest dużo prostsze niż rzeczywistym. Przeprowadzone eksperymenty potwierdzają, że podejście STP może z powodzeniem służyć do koordynacji i planowania działań podczas rzeczywistej rozgrywki.
Niewątpliwą zaletą architektury STP jest jej funkcjonalność, zachowania zawodników mogą być zmieniane na trzech poziomach w zależności od sytuacji na planszy.

Niniejsza praca dostarcza gotowe środowisko testowe wraz z aplikacją sterującą. Elementy te mogą zostać wykorzystane podczas dalszych prac nad rozgrywkami RoboCup. Stworzone oprogramowanie można
dalej rozwijać w kierunku przeprowadzenia pełnej rozgrywki wzorowanej na  \emph{Small-Size League}.