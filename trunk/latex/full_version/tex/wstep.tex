\chapter[Wstęp ]{Wstęp}
Wstęp został póki co przeklejony z pracy inżynierskiej \cite{inzynierka}.
\todo{napisać nowy wstęp}
Termin \emph{robot} został po raz pierwszy użyty w sztuce czeskiego pisarza Karela \u{C}apka. Słowo to określało maszynę-niewolnika  zastępującą człowieka w najbardziej uciążliwych zajęciach. Semantyka tego wyrazu doskonale wyraża motywację, którą kieruje się człowiek tworząc roboty. Dlatego w drugiej połowie XX wieku, kiedy technologia elektroniczna oferowała coraz większe możliwości, robotyka jako dziedzina nauki i techniki zaczęła się gwałtownie rozwijać. Początkowo konstruowano roboty do zastosowań przemysłowych. Były to manipulatory, czyli mechaniczne ramiona o kilku stopniach swobody, na których zamontowane były odpowiednie narzędzia. Pierwszy manipulator został wykorzystany w przemyśle samochodowym w 1961 roku przez firmę General Motors. 

Wraz z rozwojem techniki zaczęto myśleć o wprowadzeniu robotów do codziennego życia przeciętnego człowieka. Koncepcja robotów usługowych zakłada, że będą one wyręczać ludzi z konieczności wykonywania żmudnych czynności, takich jak sprzątanie czy też koszenie trawników. Współczesne roboty usługowe mogą nawet pełnić funkcje przewodników po muzeach. Nie wyczerpuje to oczywiście wszystkich zastosowań robotów, które coraz częściej wykorzystywane są w medycynie, wojskowości oraz ratownictwie.

Innowacyjne rozwiązania stwarzają jednak potrzebę opracowania wymagającego środowiska testowego. W przypadku robotów mobilnych gra w piłkę nożną może za takie posłużyć. Podobnie jak w przypadku ludzi zawodnik w takiej grze powinien charakteryzować się dużą zwrotnością i szybkością. Ponadto powinien sprawnie reagować na zmiany sytuacji na boisku, szybko podejmować decyzje oraz (ponieważ jest to gra zespołowa) współpracować z pozostałymi zawodnikami. Z takiego założenia wyszli twórcy \emph{RoboCup}, czyli rozgrywek robotów w piłkę nożną. Liga jest traktowana jako pole testowe dla konstrukcji mechanicznych oraz algorytmów sterowania.
W ostatnich latach stworzono także osobne przedsięwzięcie o nazwie \textit{RoboCup Rescue}, mające na celu testowanie robotów pod kątem użyteczności w ratownictwie podczas sytuacji kryzysowych.

Liga \textit{RoboCup} budzi zainteresowanie wielu ludzi na całym świecie, także liczne środowiska akademickie prowadzą prace z nią związane. Pomimo że na Politechnice Warszawskiej nie istnieje jeszcze w pełni funkcjonalna drużyna robotów, testowane są już algorytmy sterowania pojedynczym zawodnikiem.
Jednym z podstawowych problemów, na który napotyka się przy sterowaniu robotem mobilnym, jest bezkolizyjna nawigacja w dynamicznie zmieniającym się środowisku. To właśnie ten problem należy rozwiązać przed przystąpieniem do kolejnych, bardziej zaawansowanych prac.

%Autorzy niniejszej pracy postawili przed sobą dwa zasadnicze cele:
%\begin{enumerate}
%\item opracowanie środowiska symulacyjnego, odpowiadającego jednej z lig rozgrywek RoboCup, dającego w przyszłości możliwość testowania nowych rozwiązań i algorytmów kontrolujących drużynę robotów

%\item zaimplementowanie oraz przetestowanie pod kątem użyteczności w rozgrywkach RoboCup wybranego algorymu unikania kolizji  
%\end{enumerate} 

\section{Cel pracy}
Celem niniejszej pracy było stworzenie środowiska symulacyjnego, umożliwiającego modelowanie rozgrywki robotów w piłkę nożną oraz 
dającego w przyszłości możliwość testowania różnorodnych rozwiązań sterowania drużyną.

%Celem niniejszej pracy było stworzenie kompletnego środowiska symulacyjnego, pozwalającego na modelowanie robotów grających w piłkę nożną i~dającego
% w przyszłości możliwość testowania różnorodnych rozwiązań sterowania drużyną. Ponadto za zadanie przyjęto zaimplementowanie algorytmu unikania kolizji,
% który pozwalałby zawodnikowi sprawnie poruszać się w dynamicznym środowisku, jakim jest rozgrywka \textit{RoboCup}. 
%Tworząc modele robotów postawiono sobie za cel ich kompatybilność z robotami \textit{HMT} rozwijanymi aktualnie przez studenckie
% Koło Naukowe Robotyki \textit{Bionik\footnote{\url{http://bionik.ia.pw.edu.pl}}} (funkcjonujące przy Zespole Programowania Robotów i Systemów Rozpoznających 
%na Wydziale Elektroniki i Technik Informacyjnych). 
%Autorzy mają nadzieję, ze ich praca w przyszłości przyczyni się do rozwoju działań związanych z rozgrywkami robotów w piłkę nożną na naszym Wydziale, 
%a także okaże się przydatna innym osobom zajmującym się zagadnieniami związanymi z robotyką.
