\documentclass[11pt,onecolumn,a4paper,final]{article}

\usepackage[left=2.5cm,top=2.5cm,right=2.5cm,bottom=1.5cm,includehead,includefoot]{geometry}
\usepackage[MeX]{polski}
\usepackage{url}
\usepackage{graphicx}
\usepackage{float}
\usepackage[justification=centering]{caption}
\graphicspath{{./}}
\usepackage{subfig}
\usepackage{amsmath}
\usepackage{amssymb}
\usepackage{wrapfig}
\usepackage{color}
\usepackage[utf8]{inputenc}

\linespread{1.3}

\begin{document}
 

\begin{flushright}
	Warszawa, dn. 24.01.2010
\end{flushright}

\vspace{-1.45cm}

\begin{flushleft}
	Maciej Gąbka \\  nr indeksu: 198404 \\ \url{M.Gabka@stud.elka.pw.edu.pl}
\end{flushleft}

\vspace{0.5cm}

\begin{center}
	\LARGE \bfseries Plan eksperymentów do pracy magisterskiej \\
\end{center}

\vspace{1cm}

\begin{center}
\textbf{Temat pracy}: Zcentralizowany algorytm sterowania drużyną robotów w rozgrywkach RoboCup\\
\textbf{Promotor}: dr hab. in. Jarosław Arabas\
\end{center}

\section*{Eliminacje do mistrzostw świata SSL}
Każda zgłoszona do rozgrywek drużyna, przed przystąpieniem do turnieju głównego musi przejść eliminacje sprawdzające jej poziom.
Poniżej zostały zaprezentowwane wybrane problemy techniczne, których rozwiązanie było ocenianie podczas eliminacji do turnieju głównego na przestrzeni
kilku ostatnich lat\cite{robocup_www}.
\subsection*{ Nawigacja w dynamicznym środowisku (rok 2011)}
Celem próby jest sprawdzenie zdolności robotów do bezpiecznego poruszania się w dynamicznym środowisku. Poniżej zamieszczono rysunek przedstawiający 
środowisko testowe. Znajduje się na nim 6 robotów pełniących role przeszkód. Dwa z pośród nich są nieruchome, a pozostałe cztery poruszają się wzdłuż
zaznaczonej linii prostej.
\begin{figure}[!h]
\centering
\includegraphics[scale=0.3]{./test_area}
\caption{Plan środowiska testowego} \label{fig:arch}
\end{figure}
Zasady eksperymentu są następujące:
\begin{enumerate}
\item Liczba robotów jest ograniczona do trzech.
\item Uczestniczące roboty muszą poruszać sie pomiędzy dwoma nieruchomymi przeszkodami.
\item Każdorazowo kiedy robot dotknie przeszkody otrzymuje punkt ujemny.
\item Każdy uczestnik, który pokona z powodzeniem trasę otrzymuje punkt.
\item Robot, który wykona okrążenie z piłką otrzymuje 3 punkty.
\item Test trwa 2 minuty.
\end{enumerate}

\subsection*{ Mixed Team Challenge - Shooting and Passing (rok 2010)}
Podczas tego zadania w skład jednej drużyny wchodzą roboty z różnych zespołów. Zawodnicy nie mają zatem dostępu do informacji o stanie pozstałych 
uczestników meczu. Celem eksperymentu jest stworzenie scenariusza wymagającego prawdziwej kooperacji pomiędzy robotami.
Podczas rozgrywek w 2010 drużyny składały się z 4 robotów(po 2 roboty z różnych zespołów).
Każda para robotów była sterowana za pomocą własnego algorytmu. 
Zabroniono komunikacji w obrębie jednej drużyny pomiędzy zawodnikami z różnych zespołów(sterowanych przez rózne algorytmy).

Eksperyment polega na umieszczeniu robotów z dwóch różnych drużyn na boisku i zdobyciu jak największej ilości bramek w ciągu 120 sekund.
Punkty przyznawane są następująco: 
\begin{enumerate}
    \item drużyna zdobywa 1 punkt za strzelenie gola po podaniu. ( roboty z różnych podezespołów musza dotknać piłkę). 
    \item drużyna zdobywa 2 punkty gdy przed zdobyciem bramki piłka dotknie co najmniej 3 razy różnych zawodników.
\end{enumerate}
Przy czym podanie uważa się za ważne jedynie wtedy, gdy dotyczy robotów pochodzących z 2 różnych zespołów.

Zasady eksperymentu są następujące:
To start, the mixed team chooses a side of the field. All robots must be placed within 1m from their team's goal line. Opponent robots will be placed on the field as obstacles. 
A ball will be placed near one of the corners at the mixed team's own side of the field. 
A goal can be scored only when the kicking robot is in the opponent's half.
Once a goal is scored or the ball has left the field, it will be placed again near one of the corners at the own side of the field.
\begin{enumerate}
    \item  To start, all robots must be placed within 1m from the own goal line.
    
    \item  Opponent robots will be placed on the field as obstacles.
    
    \item  A ball will be placed near one of the corners at the own side of the field.
    
    \item  A goal can be scored only when the kicking robot is in the opponent's half.
    
    \item  Once a goal is scored or the ball has left the field, it will be placed again near one of the corners at the own side of the field.
\end{enumerate}

\subsection*{ Basic Challenge 1 - Shooting and Passing(rok 2009)}

Put up two to three robots on the field and score as many goals as possible within 120 seconds. Points are awarded as follows:

\begin{enumerate}
    \item  1 Point: Scoring a goal after two robots have touched the ball (i.e. after a pass)
    \item  2 Points: Scoring a goal after the ball has been touched at least three times by alternating robots (i.e. after at least two passes)
\end{enumerate}

The rules for this challenge:
\begin{enumerate}
    \item  To start, all robots must be placed within 1m from the own goal line.
    \item  Opponent robots will be placed on the field as obstacles.
    \item  A ball will be placed near one of the corners at the own side of the field.
    \item  A goal can be scored only when the kicking robot is in the opponent's half.
    \item  Once a goal is scored or the ball has left the field, it will be placed again near one of the corners at the own side of the field.
\end{enumerate}

\subsection*{Challenge 2 - Navigation 2009}
Two sets of obstacles (see attached sketch) are placed at each side of the field. One robot starts at
the center of the field and must drive from one penalty mark of the field to the other as many times as possible within 120s.
Every lap (from one set of obstacles to the other and back) counts as one point.
Each time it has to drive through the gap between two obstacles.
\begin{figure}[!h]
\centering
\includegraphics[scale=0.9]{./tcfield}
\caption{Środowisko testowe} \label{fig:arch}
\end{figure}
If it touches an obstacle, it gets a penalty of -1 points.
Additionally, teams are allowed to use up to three robots in parallel and sum up their laps.
If two robots touch each other, the penalty also applies (per touching event, not per robot, of course).
If a robot has driven at least one lap (independently of the number of collisions), the team's minimum number of points will be 1. 

\begin{thebibliography}{9}
 \bibitem{stp}
	B.Browning, J.Bruce, M.Bowling, M.Veloso:
	\emph{STP: Skills, tactics and plays for multi-robot control
             in adversarial environments.}
	  Carnegie Mellon University, 2004.

\bibitem{errt}
	J.Bruce, M.Veloso: 
	\emph{Real-Time Randomized Path Planning for Robot Navigation.}
	  Carnegie Mellon University.

\bibitem{wmasewicz}
	B.Browning, J.Bruce, M.Bowling, M.Veloso:
	\emph{Multi-Robot Team Response to a Multi-Robot
             Opponent Team.}
	  Carnegie Mellon University.
	  
\bibitem{wmasewicz}
	Oficjalna strona projektu
	\emph{www.robocup.org}

\end{thebibliography}

\end{document}
